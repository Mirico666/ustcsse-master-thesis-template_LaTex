% !TeX root = ../main.tex

\ustcsetup{
  keywords  = {分布式对象存储, 管控平台, AliIO},
  keywords* = {Distributed Object Storage, Management and Control Platform, AliIO},
}

\begin{abstract}



  随着云计算和大数据的快速发展,分布式对象存储技术已成为一种主流的数据存储方式。然而,由于分布式对象存储系统的规模庞大、复杂度高,管理和维护变得异常困难。
  因此,为了有效地管理和维护分布式对象存储系统,各大公司开发了相应的分布式对象存储系统管控平台。尽管这些平台在不断迭代的过程中具备了不少优势,但同时也存在着
  一些明显的不足,如功能不够完善、安全性差、操作复杂等,这些问题无疑增加了用户的操作难度和平台的安全风险。因此,基于解决现有系统缺陷的前提下,结合阿里自研
  分布式对象存储系统AliIO的定制化需求,开发一个功能完善、稳定高效、操作简单的分布式对象存储系统管控平台是非常有必要的。

  鉴于上述背景和需求,本文旨在研发一种基于分布式对象存储系统AliIO的管控平台,以有效管理和维护该系统。该平台采用微服务架构,包含五个核心功能模块:注册认证、权限控
  制、策略控制、文件存取和系统监控。前端采用Vue.js框架,后端采用Spring Boot框架和MySQL数据库,以确保系统的高效性。在认证模块方面,平台利用OAuth2和JWT实现安全认
  证和授权机制。在权限控制模块中,采用SpringCloud Gateway作为API网关,对用户请求进行鉴权控制。在系统监控模块中,通过Prometheus监控工具采集和存储对象存储服务器相
  关状态数据,并通过Grafana数据可视化工具展示监控数据。此外,本文从功能测试、性能测试、安全性测试和易用性测试四个方面对该管控平台进行全面细致的测试,测试结果表明
  该系统能够满足正常的功能需求,在性能、安全性和易用性方面也达到了良好的效果。综上所述,该管控平台的研发取得了理想的结果,达到了设计目标。

  本文研究和实现的分布式对象存储系统管控平台能够有效地管理和控制分布式对象存储系统,该平台实现了对权限控制、对象存储、用户和安全等方面的全面管理,为分布式对象存储
  系统的管理和控制提供了一种有效的解决方案。在未来的工作中,将继续对该管控平台进行优化和改进,提高平台的性能和效率,以满足更多场景下的需求。


  % 摘要分中文和英文两种,中文在前,英文在后,博士论文中文摘要一般 800~1500 个汉字,硕士论文中文摘要一般 500~1000 个汉字。
  % 英文摘要的篇幅参照中文摘要。

  % 关键词另起一行并隔行排列于摘要下方,左顶格,中文关键词间空一字或用分号“,”隔开,英文关键词之间用逗号“,”或分号“;”隔开。

  % 中文摘要是论文内容的总结概括,应简要说明论文的研究目的、基本研究内容、研究方法或过程、结果和结论,突出论文的创新之处。
  % 摘要应具有独立性和自明性,即不用阅读全文,就能获得论文必要的信息。
  % 摘要中不宜使用公式、图表,不引用文献。

  % 中文关键词是为了文献标引工作从论文中选取出来用以表示全文主题内容信息的单词和术语,一般 3~8 个词,要求能够准确概括论文的核心内容。
\end{abstract}

\begin{abstract*}
  % This is a sample document of USTC thesis \LaTeX{} template for bachelor,
  % master and doctor. The template is created by zepinglee and seisman, which
  % orignate from the template created by ywg. The template meets the
  % equirements of USTC thesis writing standards.

  % This document will show the usage of basic commands provided by \LaTeX{} and
  % some features provided by the template. For more information, please refer to
  % the template document ustcthesis.pdf.

  With the rapid development of cloud computing and big data, distributed object storage technology has become a mainstream data storage method. 
  However, due to the large scale and high complexity of the distributed object storage system, management and maintenance become extremely 
  difficult. Therefore, in order to effectively manage and maintain distributed object storage systems, major companies have developed 
  corresponding distributed object storage system management and control platforms. Although these platforms have many advantages in the 
  process of continuous iteration, there are also some obvious shortcomings, such as insufficient functions, poor security, and complicated 
  operations. These problems undoubtedly increase the difficulty of user operations and the security risks of the platform. . Therefore, based on 
  the premise of solving the defects of the existing system, combined with the customization requirements of AliIO, Ali's self-developed distributed
  object storage system, it is very necessary to develop a distributed object storage system management and control platform with complete 
  functions, stable and efficient, and simple operation.

  In view of the above background and requirements, this paper aims to develop a management and control platform based on the distributed object 
  storage system AliIO to effectively manage and maintain the system. The platform adopts micro-service architecture and includes five core 
  functional modules: registration authentication, authority control, policy control, file access and system monitoring. The front-end adopts 
  the Vue.js framework, and the back-end adopts the Spring Boot framework and MySQL database to ensure the efficiency of the system. In terms 
  of authentication modules, the platform uses OAuth2 and JWT to implement security authentication and authorization mechanisms. In the authority 
  control module, SpringCloud Gateway is used as the API gateway to perform authentication control on user requests. In the system monitoring 
  module, the Prometheus monitoring tool is used to collect and store the relevant status data of the object storage server, and the monitoring 
  data is displayed through the Grafana data visualization tool. In addition, this paper conducts a comprehensive and detailed test on the 
  management and control platform from four aspects: function test, performance test, security test and usability test. The test results show that 
  the system can meet the normal functional requirements. Good results have also been achieved in terms of usability. To sum up, the research and 
  development of the management and control platform has achieved ideal results and achieved the design goal.

  The distributed object storage system management and control platform researched and implemented in this paper can effectively manage and 
  control the distributed object storage system. management and control provides an effective solution. In the future work, we will continue 
  to optimize and improve the management and control platform to improve the performance and efficiency of the platform to meet the needs of 
  more scenarios.
\end{abstract*}

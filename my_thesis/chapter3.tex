\chapter{需求分析}
% 本章首先对分布式对象存储管理系统进行整体功能上的概述, 从功能性和非功能性两个维度分析分布式对象存储管理系统的整体需求。
% 需求分析是分布式对象存储管理系统开发的关键步骤, 做好需求分析工作有助于接下来的工作顺利进行。

本文首次对分布式对象内存管理系统作出了整体功能上的概括,从功能性与非功能性二种角度剖析了分布式对象内存管理系统的总体要
求。需求分析工作是分布式对象存储管理系统软件研发的关键步骤,正确进行需求分析工作将有利于以后的工作顺利进行。

\section{功能性需求}
%系统用例图
分布式对象存储管理系统软件,是集文件存放功能与应用/系统管理为一身的综合系统。本系统目前主要有二种使用者,一类是存储应用的人员,另一类则是系统管理人员。

存储使用人员为整个分布式对象存储的直接服务群体,他们注册属于自己的账号之后,可以直接创建自己的Bucket,在每个Bucket下上传、下载文件(Object),支持多种文件类型,
用户可对自己的Bucket进行管理,也支持多文件用时上传和下载。用户可对Bucket进行创建、删除(Bucket为空时)、重命名以及访问策略的设置。

系统管理员工主要负责客户数据、系统分组、系统日志、系统状态、存取方式和集群扩容管理,因此他们还必须对已登录注册的客户和系统的分组进行控制。这类应用必须注册登记到系统管理再完成所有功能使用。通常,系统管理员工注册到系统管理时首先会检查用户列表、分组情况、存储使用情况和系统状态等。还应特别关注系统的告警信息,对系统进行及时的修复。对于用户访问策略以及分组
的访问策略也必须由系统管理员负责,主要是为了资源的隔离,用户的最终访问策略由用户个人访问策略和分组访问策略共同决定,即用户最终访问策略=用户访问策略+分组访问策略。

系统中的策略主要分为Bucket访问策略和用户访问策略。Bucket访问策略主要有public、custom和private三种,系统内置的用户访问策略主要有五种,分别是控制台管理员(consoleAdmin)策略、
诊断(diagnostics)策略、只读(readonly)策略、读写(readwrite)策略和只写(writeonly)策略。当设置Bucket的使用方式是public后,用户必须不通过任何验证才能进行使用资源;当设置Bucket的访问方式是custom后,Bucket的最终访问策略由具体的用户访问策略决定,用户访问策略可以是系统内定的策略,如只读(readonly)策略、读写(readwrite)策略和只写(writeonly)策略,
也可以是自定义的访问策略;当设置Bucket的访问策略为private时,用户未经授权不能进行任何操作,所有用户访问策略失效。因此,Bucket的访问策略权限要大于用户访问策略。

系统中暂时只有三个人物:特级经理,一般经理,还有一般用户。系统管理员也是目前唯一的超级管理者,掌握对操作系统中各种功能的运行权限,重点是能够检测服务器状态并且对系统进行修改或者扩展;一般管理者,掌握对使用者、分组和访问策略的管理权限,如查看用户信息和分组信息,对用户或分组进行禁用,修改用户或分组的访问策略;普通用户拥有对Bucket管理和文件上传下载
等权限。

% 系统中的权限指微服务接口的访问权限行,一条功能接口对应着一条权限。

% 通过将上面的二类应用需求分类,即可得到应用需求列表。



% \begin{table}[htbp]
%     \caption{用户需求表}
%     \centering
%     \renewcommand\arraystretch{1.5}{
%     \setlength{\tabcolsep}{7mm}{
%     % \begin{tabular}{|c|c|c|}
%     \begin{tabular}{|p{2cm}<{\centering}|p{3cm}<{\centering}|p{5cm}<{\centering}|}
%     \hline
%     \bf{需求编号} & \bf{需求名称} & \bf{需求说明} \\
%     \hline
%     R01 & 用户注册 & 所有人员需要注册登录账号 \\
%     \hline
%     R02 & 用户登录 & 所有人员登录系统后才能进行各种操作 \\
%     \hline
%     R03 & 增加角色 & 系统管理员增加角色 \\
%     \hline
%     R04 & 删除角色 & 系统管理员删除角色 \\
%     \hline
%     R05 & 查看角色 & 系统管理员查看角色 \\
%     \hline
%     R06 & 修改角色 & 系统管理员修改角色 \\
%     \hline
%     R07 & 增加用户 & 系统管理员增加用户 \\
%     \hline
%     R08 & 删除用户 & 系统管理员删除用户 \\
%     \hline
%     R09 & 查看用户 & 系统管理员查看用户 \\
%     \hline
%     R10 & 修改用户 & 系统管理员修改用户 \\
%     \hline
%     R11 & 增加分组 & 系统管理员增加分组 \\
%     \hline
%     R12 & 删除分组 & 系统管理员删除分组 \\
%     \hline
%     R13 & 查看分组 & 系统管理员查看分组 \\
%     \hline
%     R14 & 修改分组 & 系统管理员修改分组 \\
%     \hline
%     R15 & 增加策略 & 系统管理员增加策略 \\
%     \hline
%     R16 & 删除策略 & 系统管理员删除策略 \\
%     \hline
%     R17 & 查看策略 & 系统管理员查看策略 \\
%     \hline
%     R18 & 修改策略 & 系统管理员修改策略 \\
%     \hline
%     R19 & 增加权限 & 系统管理员增加权限 \\
%     \hline
%     R20 & 删除权限 & 系统管理员删除权限 \\
%     \hline
%     R21 & 查看权限 & 系统管理员查看权限 \\
%     \hline
%     R22 & 修改权限 & 系统管理员修改权限 \\
%     \hline
%     R23 & 角色分配 & 系统管理员给用户分配对应的角色 \\
%     \hline
%     R24 & 组分配 & 系统管理员给用户分配对应的组 \\
%     \hline
%     R25 & 策略分配 & 系统管理员给用户分配对应的策略 \\
%     \hline
%     R26 & 权限分配 & 系统管理员给用户分配对应的权限 \\
%     \hline
%     R27 & 动态添加网关路由 & 在API网关中动态添加路由信息 \\
%     \hline
%     R28 & 服务器控制 & 系统管理员启动和停止服务器 \\
%     \hline
%     R29 & 服务器修复 & 系统管理员进行服务器修复 \\
%     \hline
%     R30 & 查看服务器状态 & 系统管理员查看服务器状态 \\
%     \hline
%     R31 & 集群扩容 & 系统管理员进行集群扩容 \\
%     \hline
%     R32 & 集群清理 & 系统管理员进行集群清理 \\
%     \hline
%     R33 & 查看集群状态 & 系统管理员查看集群状态 \\
%     \hline
%     R34 & 查看集群状态 & 系统管理员查看集群状态 \\
%     \hline
%     \end{tabular}}}
% \end{table}

\begin{center}
    \renewcommand\arraystretch{1.5}{
    \setlength{\tabcolsep}{5mm}{
	\begin{longtable}{|p{2.2cm}<{\centering}|p{3.4cm}<{\centering}|p{5.6cm}<{\centering}|}
		\caption{用户需求表}\label{用户需求表}\\
		\hline
        \bf{需求编号} & \bf{需求名称} & \bf{需求说明} \\
        \hline
        R01 & 用户注册 & 存储使用人员需要注册登录账号 \\
        \hline
        R02 & 用户登录 & 所有人员登录系统后才能进行各种操作 \\
        \hline
        R03 & 增加角色 & 系统管理员增加角色 \\
        \hline
        R04 & 删除角色 & 系统管理员删除角色 \\
        \hline
        R05 & 查看角色 & 系统管理员查看角色 \\
        \hline
        R06 & 修改角色 & 系统管理员修改角色 \\
        \hline
        R07 & 增加用户 & 系统管理员增加用户 \\
        \hline
        R08 & 删除用户 & 系统管理员删除用户 \\
        \hline
        R09 & 查看用户 & 系统管理员查看用户 \\
        \hline
        R10 & 修改用户 & 系统管理员修改用户 \\
        \hline
        R11 & 增加分组 & 系统管理员增加分组 \\
        \hline
        R12 & 删除分组 & 系统管理员删除分组 \\
        \hline
        R13 & 查看分组 & 系统管理员查看分组 \\
        \hline
        R14 & 修改分组 & 系统管理员修改分组 \\
        \hline
        R15 & 增加策略 & 系统管理员增加策略 \\
        \hline
        R16 & 删除策略 & 系统管理员删除策略 \\
        \hline
        R17 & 查看策略 & 系统管理员查看策略 \\
        \hline
        R18 & 修改策略 & 系统管理员修改策略 \\
        \hline
        R19 & 增加权限 & 系统管理员增加权限 \\
        \hline
        R20 & 删除权限 & 系统管理员删除权限 \\
        \hline
        R21 & 查看权限 & 系统管理员查看权限 \\
        \hline
        R22 & 修改权限 & 系统管理员修改权限 \\
        \hline
        R23 & 角色分配 & 系统管理员给用户分配对应的角色 \\
        \hline
        R24 & 组分配 & 系统管理员给用户分配对应的组 \\
        \hline
        R25 & 策略分配 & 系统管理员给用户和组分配对应的策略 \\
        \hline
        R26 & 权限分配 & 系统管理员给用户分配对应的权限 \\
        \hline
        R27 & 动态添加网关路由 & 在API网关中动态添加路由信息 \\
        \hline
        R28 & 服务器控制 & 超级管理员启动和停止服务器 \\
        \hline
        R29 & 服务器修复 & 超级管理员进行服务器修复 \\
        \hline
        R30 & 查看服务器信息 & 超级管理员查看服务器信息 \\
        \hline
        R31 & 集群扩容 & 超级管理员进行集群扩容 \\
        \hline
        R32 & 集群清理 & 超级管理员进行集群清理 \\
        \hline
        R33 & 查看集群信息 & 超级管理员查看集群信息 \\
        \hline
        R34 & 查看日志信息 & 系统管理员查看系统的日志信息 \\
        \hline
        R35 & 查看告警信息 & 超级管理员查看系统的告警信息 \\
        \hline
        R36 & 创建Bucket & 系统管理员和存储使用人员增加Bucket \\
        \hline
        R37 & 删除Bucket & 系统管理员和存储使用人员删除Bucket \\
        \hline
        R38 & 查看Bucket & 系统管理员和存储使用人员查看Bucket \\
        \hline
        R39 & 修改Bucket & 系统管理员和存储使用人员修改Bucket \\
        \hline
        R40 & 文件上传 & 存储使用人员进行文件上传 \\
        \hline
        R41 & 文件下载 & 存储使用人员进行文件下载 \\
        \hline
	\end{longtable}}}
\end{center}

网络管理者分为特级管理者与一般管理者,其中特级管理者享有最高级别的权力,而一般管理者则享有次高等权力,二者都可以控制其他
的所有用户和角色资金的分配,而特级管理者还享有对主机和集群资金的控制权力。超级管理员用户在系统最初运行的时候会被创建好。
在普通管理人员第一次登录系统的时候,就必须完成登录,因为该系统并不能允许外部的管理人员进行登录管理员工作账号。登录时,必
须选择登录帐号的所有角色名和与系统管理员协商的对应密钥,当后台对密钥校验通过后,普通管理员的相关信息就会被加载进去。对于
存储使用的人员,第一次登录系统同样要登录用户名,登录时,必须选择登录帐户的所属角色,并填报账号、密码、电子邮箱等个人信息,
在后台对个人信息查重校核通过后,才会增加用户帐号的相关信息。

根据前文中提到的系统需求图三点一和系统用例图三点一,即可获取系统用例图三点二,而系统管理员和超级管理者的使用范围则包括图
三点二的全部信息,而普通管理者使用范围则包括图三点二中除了服务器和集群管理系统之外的全部信息,存储的系统用例则包括用户、
登陆、Bucket管理、文档提交、软件下载。

\begin{center}
    \renewcommand\arraystretch{1.5}{
    \setlength{\tabcolsep}{5mm}{
	\begin{longtable}{|p{2.2cm}<{\centering}|p{3.4cm}<{\centering}|p{5.6cm}<{\centering}|}
		\caption{系统用例表}\label{系统用例表}\\
		\hline
        \bf{用例编号} & \bf{用例名称} & \bf{涉及的需求编号} \\
        \hline
        UC01 & 注册 & R01 \\
        \hline
        UC02 & 登录 & R02 \\
        \hline
        UC03 & 角色管理 & R03,R04,R05,R06 \\
        \hline
        UC04 & 用户管理 & R07,R08,R09,R10 \\
        \hline
        UC05 & 分组管理 & R11,R12,R13,R14 \\
        \hline
        UC06 & 策略管理 & R15,R16,R17,R18 \\
        \hline
        UC07 & 权限管理 & R19,R20,R21,R22 \\
        \hline
        UC08 & 用户角色分配 & R23 \\
        \hline
        UC09 & 用户组分配 & R24 \\
        \hline
        UC10 & 用户策略分配 & R25 \\
        \hline
        UC11 & 组策略分配 & R25 \\
        \hline
        UC12 & 角色权限分配 & R26 \\
        \hline
        UC13 & 网关路由管理 & R27 \\
        \hline
        UC14 & 服务器管理 & R28,R29,R30,R35 \\
        \hline
        UC15 & 集群管理 & R31,R32,R33 \\
        \hline
        UC16 & 日志管理 & R34 \\
        \hline
        UC17 & Bucket管理 & R36,R37,R38,R39 \\
        \hline
        UC18 & 文件上传 & R40 \\
        \hline
        UC19 & 文件下载 & R41 \\
        \hline
	\end{longtable}}}
\end{center}

表\ref{系统用例表}简要列举了所有的系统用例, 接下来对这些系统用例进行详细描述与分析。

(1) 注册申请用例

% 注册申请用例表如表\ref{注册申请用例表}所示。需要注册的用户主要包括普通管理员和存储使用人员,对于普通管理员,由于该系统的管理权限只提供给公司内部使用,不希望外部人员随意注册管理员
% 账号,因此在注册时有严格的要求。对于想要获得管理员权限的公司内部人员,超级管理员会与其协商注册密钥,一个角色对应一个密钥,系统中暂时只有普通管理员和普通用户这两个角
% 色,用户注册时需要将自己希望创建的角色以及对应的密钥在前端进行选择,之后再对用户名、密码以及账号描述等账号有关的信息进行填写。后台会在用户创建成功后给超级管理员和公
% 司企业邮箱发送邮件,告知用户创建成功,公司和超级管理员可对账号信息进行手动核实,进一步保障系统安全。

需要登录的普通用户主要分为一般管理者和企业存储内部应用人员,对于一般管理者,因为该操作系统的管理权限仅提供给企业内部应用人员,不希望外界
人士擅自登录管理者帐户,所以在登录时都会有很严格的规定。而对于希望获取管理员权限的企业内部人员,超级管理员也会与其协商申
请密钥,每一个人物都对应着一种密钥,操作系统中暂时只有一般管理者和一般使用者这二种人物,使用者在注册时必须先将自己想要创
建的人物及其相应的密钥在前台进行选择,然后再对账号、密钥和账号描述以及与账号相关的信息进行选择。后台会在用户建立完成后
向超级管理员和公司企业邮箱发送电子邮件,告知使用者建立已完成,公司企业和超级管理员也可对帐号信息进行自动核对,以进一步保
证安全。

对于普通用户,限制没有那么严格,只需注册的时候填写正确的用户名、密码、邮箱等信息即可,保证用户名不重复且为英文即可通过验证,成功注册平台账户。在用户注册平台账号的
同时,AliIO对象存储服务会根据用户输入的账号和密码自动生成AccessKey 和 SecretKey,这样就生成了对应的存储服务器上的用户,AliIO对象存储服务会自动为该用户分配存储空间。
用户的用户名、密码、邮箱、空间使用情况等信息均在存储在Mysql数据库中。

\begin{center}
    \renewcommand\arraystretch{1.5}{
    \setlength{\tabcolsep}{8mm}{
	\begin{longtable}{|p{2.2cm}<{\centering}|p{9cm}|}
		\caption{注册申请用例表}\label{注册申请用例表}\\
        \hline
        \bf{用例编号} & UC01 \\
        \hline
        \bf{用例名称} & 注册申请 \\
		\hline
        \bf{用例参与者} & 普通管理员、普通用户 \\
		\hline
        \bf{前置条件} & 该普通管理员或普通用户尚未注册 \\
		\hline
        \bf{触发条件} & 由普通用户或普通用户通过在网页上填写用户资料或提交数据的后置条件帐号创建并完成,根据用户资料自动关联的角色信息,给对应邮
        箱发送账号创建的成功邮件提醒 \\
		\hline
        \bf{后置条件} & 账号创建成功,给对应邮箱发送账号创建成功邮件提醒 \\
		\hline
        \bf{正常流程} 
        & 1.普通管理员或普通用户根据要求选择需要创建的角色,普通管理员填写对应的密钥。\\
        & 2.普通管理员或普通用户填写了基本的帐号信息。\\
        & 3.填写验证码。\\
        & 4.后台校验通过后自动创建账号以及关联角色。普通用户注册还会生成AccessKey 和 SecretKey生成对应的存储服务器用户。\\
        & 5.后台将用户成功创建的信息邮件发送出去。\\
		\hline
        \bf{异常流程} 
        & 1.填写的用户名非全英文字母或已存在,提示用户名格式错误或用户名已存在。\\
        & 2.普通管理员填写的密钥不正确,提示密钥错误。\\
        & 3.用户注册时填写的密码太短或太长或全为数字,提示用户重新填写符合要求的密码。\\
        & 4.确认密码时,重新输入的密码不一致,提示密码不一致,重新输入。\\
        & 5.验证码输入错误,提示用户刷新验证码并重新输入。\\
        
		\hline
	\end{longtable}}}
\end{center}

(2) 登录用例

登录用例如表3.4所示。 系统管理员和普通用户都需要登录用例, 
普通管理者和普通用户必须在注册完成后,才能进入系统注册界面完成登记。但如果普通用户忘记了密码, 可通过邮箱认证后
进行修改。

\begin{center}
    \renewcommand\arraystretch{1.5}{
    \setlength{\tabcolsep}{8mm}{
	\begin{longtable}{|p{2.2cm}<{\centering}|p{9cm}|}
		\caption{登录用例表}\label{登录用例表}\\
        \hline
        \bf{用例编号} & UC02 \\
        \hline
        \bf{用例名称} & 登录用例 \\
		\hline
        \bf{用例参与者} & 系统管理员、普通用户 \\
		\hline
        \bf{前置条件} & 用户已成功注册账号并且尚未登录 \\
		\hline
        \bf{触发条件} & 用户在Web前端点击登录按钮进入登录页面 \\
		\hline
        \bf{后置条件} & 用户成功登入管理系统 \\
		\hline
        \bf{正常流程} 
        & 1.用户正常启动系统的登录界面并完成登录。\\
        & 2.用户根据界面提示输入用户名和密码。\\
        & 3.系统将信息请求发送给服务器,而服务器在接收到输入的信息后立即完成了确认。\\
        & 4.登陆完成后,系统将切换至系统管理首页。\\
		\hline
        \bf{异常流程} 
        & 1.用户输入错误的用户名。\\
        & 2.用户输入错误的密码。\\
        
		\hline
	\end{longtable}}}
\end{center}

(3) 角色管理用例

% 角色管理用例如表\ref{角色管理用例表}所示。角色管理用例包含创建角色、删除角色、 修改角色与查找角色,
% 此用例只与角色的操作有关,与用户和权限的联动操作无关。
% 目前系统中存在三种角色,随着系统的迭代和需求变更,可能会添加新的角色或修改、删除原有的角色。

角色管理用例如下表三点五中所示。角色管理的使用案例主要包括创建人物、清除人物、更改人物和寻找新角色等,其使用案例通常
仅与人物的使用相关,与用户功能和权限的联动使用无关。而目前系统中存在三种角色,但由于系统的迭代和需要变化,可能会增加新
的人物或更改、清除原来的人物。

\begin{center}
    \renewcommand\arraystretch{1.5}{
    \setlength{\tabcolsep}{8mm}{
	\begin{longtable}{|p{2.2cm}<{\centering}|p{9cm}|}
		\caption{角色管理用例表}\label{角色管理用例表}\\
        \hline
        \bf{用例编号} & UC03 \\
        \hline
        \bf{用例名称} & 角色管理用例 \\
		\hline
        \bf{用例参与者} & 超级管理员\\
		\hline
        \bf{前置条件} & 用户登录后进入权限管理 \\
		\hline
        \bf{触发条件} & 超级管理员对 \\
		\hline
        \bf{后置条件} & 超级管理员对角除成功 \\
		\hline
        \bf{正常流程} 
        & 1.超级管理。\\
        & 2.系统对用户身份进行验证, 确认为超级管理员后才可继续操作。\\
        & 3.系统对角色列表进行展示。\\
        & 4.超级管理员对。\\
		\hline
        \bf{异常流程} 
        & 1.普通管理员或普通用户登录系统。\\
        & 2.尝试对超级管理员这一色。\\
        
		\hline
	\end{longtable}}}
\end{center}

(4) 用户管理用例

% 用户管理用例如表\ref{用户管理用例表}所示。用户管理用例负责管理所有普通用户的账号,这个账号包括管理系统账号和
% 存储系统账号,系统管理员可以增加用户、删除用户、修改用户和查找用户,过程类似于角色管理。用户进行账号注册后就会自动创建管理系统用户信息和存储系统用户信息,二者的用户名和密码相同,
% 存储系统用户信息会包含与存储使用情况相关的信息,这些信息是实时获取的,每次都是从存储服务器返回的最新值。管理系统用户信息会永久性的保存在数据库中。一般情况下,不会主动用到该功能模块,
% 主要用于用户忘记密码、角色密钥谢了或外部人员非法创建了用户等特殊情况,此时可以更改密码或删除用户。

客户管理用如表三点六中所示。客户管理用例负责管理每个普通用户的帐号,而这个帐户又分为管理帐号和储备机制帐户,系统经营者
还能够添加用户、撤销客户、修改客户和查找新客户,过程类似于角色管理。当客户进行帐号登录后,就会自行产生管理用户信息和储
备机制用户信息,两者的管理账号和密码都一样,而储备机制用户信息则会包括任何和存储用户情况有关的个人资料,这些信息是实时
获取的,每次都是从存储服务器返回的最新值。管理系统用户信息会永久性的保存在数据库中。通常情况下,无法主动使用的功能模块,
主要用于玩家遗忘帐号、角色密钥谢用或外部人私自建立新帐户等特定状态,此时才能修改帐号或撤销账号。

\begin{center}
    \renewcommand\arraystretch{1.5}{
    \setlength{\tabcolsep}{8mm}{
	\begin{longtable}{|p{2.2cm}<{\centering}|p{9cm}|}
		\caption{用户管理用例表}\label{用户管理用例表}\\
        \hline
        \bf{用例编号} & UC04 \\
        \hline
        \bf{用例名称} & 用户管理用例 \\
		\hline
        \bf{用例参与者} & 系统管理员\\
		\hline
        \bf{前置条件} & 用户登用户管理按钮 \\
		\hline
        \bf{触发条件} & 系统管理员对用户或删除操作 \\
		\hline
        \bf{后置条件} & 系统管理员、修作成功 \\
		\hline
        \bf{正常流程} 
        & 1.由系统管理员注册进入到系统管理网页,并单击系统管理按键。\\
        & 2.对客户身份进行校验,确定是系统管理员后才可以继续运行。\\
        & 3.系统对用户列表进行展示。\\
        & 4.系统管理员对用操作。\\
		\hline
        \bf{异常流程} 
        & 1.普通用户登录系统,提示没有对用户进行管理的权限。\\
        & 2.尝试除。\\ 
		\hline
	\end{longtable}}}
\end{center}
(5) 分组管理用例

分组管理用例如表\ref{分组管理用例表}所示。分组管理用例负责管理为存储系统用户分配的分组,系统管理员可以创建分组、删除分组、修改分组和查找分组,其过程类似于用户管理。
此用例只与分组的操作有关,与用户和分组、分组和策略的联动操作无关。这里的对分组的操作直接通过存储服务器实现,不会保存在数据库中。

\begin{center}
    \renewcommand\arraystretch{1.5}{
    \setlength{\tabcolsep}{8mm}{
	\begin{longtable}{|p{2.2cm}<{\centering}|p{9cm}|}
		\caption{分组管理用例表}\label{分组管理用例表}\\
        \hline
        \bf{用例编号} & UC05 \\
        \hline
        \bf{用例名称} & 分组管理用例 \\
		\hline
        \bf{用例参与者} & 系统管理员\\
		\hline
        \bf{前置条件} & 系统用户在注册后进行策略管理,后单击用户管理按钮触发条件系统经营者对分组进行查询、添加、更改或移除<br>后置条件系统经营者对分组进行查询、
        增加、更改或取消成功 \\
		\hline
        \bf{触发条件} & 系统管理改或删除 \\
		\hline
        \bf{后置条件} & 系统管理员对除成功 \\
		\hline
        \bf{正常流程} 
        & 1.由系统管理员用户注册加入系统管理网页,并单击分组系统管理按键。\\
        & 2.先对系统身份进行校验,在确定是系统管理员后才可以继续运行。\\
        & 3.系统对分组列表进行展示。\\
        & 4.系统管理员对分组执行查看、增加、修改或删除等操作。\\
		\hline
        \bf{异常流程} 
        & 1.普通用户登录系统,提示没有对分组进行管理的权限。\\
        & 2.当尝试清除系统管理员分组,但系统显示为不能清除。\\  
		\hline
	\end{longtable}}}
\end{center}

(6) 策略管理用例

策略管理用例如表\ref{策略管理用例表}所示。策略管理使用案例负责系统中所有的访问策略,而系统中的访问策略主要是指定普通用户对所有Bucket的访问权限,Bucket访问策略主要有public、custom和private
三种,系统内置的用户访问策略主要有五种,分别是控制台管理员(consoleAdmin)策略、诊断(diagnostics)策略、只读(readonly)策略、读写(readwrite)策略和只写
(writeonly)策略。根据需求,系统管理可以创建自定义的策略、删除自定义策略、修改自定义策略和查看所有策略。

\begin{center}
    \renewcommand\arraystretch{1.5}{
    \setlength{\tabcolsep}{8mm}{
	\begin{longtable}{|p{2.2cm}<{\centering}|p{9cm}|}
		\caption{策略管理用例表}\label{策略管理用例表}\\
        \hline
        \bf{用例编号} & UC06 \\
        \hline
        \bf{用例名称} & 策略管理用例 \\
		\hline
        \bf{用例参与者} & 系统管理员\\
		\hline
        \bf{前置条件} & 用户登录后进入策略管理系统后点击策略管理按钮 \\
		\hline
        \bf{触发条件} & 系统管理员对所有策略进行查看,对自定义策略进行创建、修改或删除 \\
		\hline
        \bf{后置条件} & 系统管理员对所有策略进行查看,对自定义策略进行创建、修改或删除成功 \\
		\hline
        \bf{正常流程} 
        & 1.系统管理员登陆后进入系统管理网站,并选择策略管理。\\
        & 2.系统对用户身份进行验证, 确认为系统管理员后才可继续操作。\\
        & 3.系统对策略列表进行展示。\\
        & 4.系统管理员查看所有策略,创建、修改或删除自定义策略。\\
		\hline
        \bf{异常流程} 
        & 1.普通用户登录系统,提示没有对策略进行管理的权限。\\
        & 2.尝试删除、修改系统内置的策略,系统提示无法删除。\\  
		\hline
	\end{longtable}}}
\end{center}

(7) 权限管理用例

权限管理用例如表\ref{权限管理用例表}所示。授权控制用例主要是对系统中的各种授权方式进行控制,而这种授权一般是指针对
微功能接口的访问授权,尽管每个功能模块中都对外暴露了访问接口,但只有具备接口授权的系统才可以使用对应的功能模块,例如
文档管理功能中就暴露了Bucket管理接口,以及上传文档与下载软件管理接口。
% 权限管理用例负责对管理系统中的所有权限进行管理,这些权限主要是针对微服务接口的访问权限,
% 虽然所有功能模块都对外暴露了访问接口,但只有拥有接口权限的
% 用户才可访问相应的功能模块,如文件管理模块中暴露了Bucket管理接口,上传文件和下载文件接口。

\begin{center}
    \renewcommand\arraystretch{1.5}{
    \setlength{\tabcolsep}{8mm}{
	\begin{longtable}{|p{2.2cm}<{\centering}|p{9cm}|}
		\caption{权限管理用例表}\label{权限管理用例表}\\
        \hline
        \bf{用例编号} & UC07 \\
        \hline
        \bf{用例名称} & 权限管理用例 \\
		\hline
        \bf{用例参与者} & 超级管理员\\
		\hline
        \bf{前置条件} & 用户在注册后进入权限管理,后单击授权管理按钮触发条件超级管理员对策略进行查询、建立、更改或移除<br>后置条件超级管理者对策略进行查询、
        创建、更改或取消成功 \\
		\hline
        \bf{触发条件} & 超级管理员对策略进行查看、创建、修改或删除 \\
		\hline
        \bf{后置条件} & 超级管理员对策略进行查看、创建、修改或删除成功 \\
		\hline
        \bf{正常流程} 
        & 1.超级用户登陆进入系统管理界面,并单击权限控制按钮。\\
        & 2.系统对用户身份信息进行了校验,确定是超级管理员后才可以继续运行。\\
        & 3.系统对权限列表进行展示。\\
        & 4.超级用户对权限进行创建、更改及撤销功能。\\
		\hline
        \bf{异常流程} 
        & 1.普通管理员或普通用户登录系统,提示没有对权限进行管理的权限。\\
        & 2.尝试取消权限控制模块的权限,但系统显示为没有删除。\\  
		\hline
	\end{longtable}}}
\end{center}

(8) 用户角色分配用例

% 用户角色分配用例如表所示。在用户注册帐号后,由于操作系统会在后台对帐号和对应的人物实行绑定,所以,在通常情况下也不要主动
% 去使用该功能,
主要等系统管理员在出现特定状态后加以解决。而人物分配的方法则是当超级用户选定了某个人物后,所有的用户信息
都将在前端界面展现出来,即可同时选择多种客户名称,而任何客户也一旦选定了可以与其他用户绑定的人物。

% 当用户注册账号时,系统会在后端将账号与相应的角色进行绑定,因此,一般情况下也不会主动去使用该模块,主要是系统管理员在
% 遇到特殊情况时进行处理。
% 角色分配的方式是超级管理员选择一个角色时,所有的用户信息会在前端页面显示出来,可以同时选定多个用户信息,任意用户被选
% 中即可为该用户绑定该角色。

\begin{center}
    \renewcommand\arraystretch{1.5}{
    \setlength{\tabcolsep}{8mm}{
	\begin{longtable}{|p{2.2cm}<{\centering}|p{9cm}|}
		\caption{用户角色分配用例表}\label{用户角色分配用例表}\\
        \hline
        \bf{用例编号} & UC08 \\
        \hline
        \bf{用例名称} & 用户角色分配用例 \\
		\hline
        \bf{用例参与者} & 超级管理员\\
		\hline
        \bf{前置条件} & 用户登录后进入权限管理系统后点击用户角色分配按钮 \\
		\hline
        \bf{触发条件} & 超级管理员选中指定角色后绑定用户 \\
		\hline
        \bf{后置条件} & 用户角色成功分配 \\
		\hline
        \bf{正常流程} 
        & 1.超级用户可以登录进入用户角色设置界面。\\
        & 2.超级管理员选中一个角色。\\
        & 3.超级管理员为该角色分配用户。\\
		\hline
        \bf{异常流程} 
        & 无\\
		\hline
	\end{longtable}}}
\end{center}


(9) 用户组分配用例

用户组分配用例如表\ref{用户组分配用例表}所示。用户组分配是在系统管理员创建分组后进行分配,这里的分组主要是针对存储系
统账户进行分组,分在同一组的用户会共享组的访问策略。组分配的基本方法是当系统管理员选择一个组时,所有的用户信息都会在前
端界面展现出来,可以同时选择多种用户信息,而任何用户也被选择为可以与该用户信息绑定的该分组。

% 组分配的方式是系统管理员选择一个组时,所有的用户信息会在前端页面显示出来,可以同时选定多个用户信息,任意用户被选中即
% 可为该用户绑定该分组。

\begin{center}
    \renewcommand\arraystretch{1.5}{
    \setlength{\tabcolsep}{8mm}{
	\begin{longtable}{|p{2.2cm}<{\centering}|p{9cm}|}
		\caption{用户组分配用例表}\label{用户组分配用例表}\\
        \hline
        \bf{用例编号} & UC09 \\
        \hline
        \bf{用例名称} & 用户组分配用例 \\
		\hline
        \bf{用例参与者} & 系统管理员\\
		\hline
        \bf{前置条件} & 用户登录后进入策略管理系统后点击用户组分配按钮 \\
		\hline
        \bf{触发条件} & 系统管理员选中指定组后绑定用户 \\
		\hline
        \bf{后置条件} & 用户成功绑定分组 \\
		\hline
        \bf{正常流程} 
        & 1.系统管理者只能加入到用户组的页面。\\
        & 2.系统管理员选中一个分组。\\
        & 3.系统管理员为该分组分配用户。\\
		\hline
        \bf{异常流程} 
        & 无\\
		\hline
	\end{longtable}}}
\end{center}

(10) 用户策略分配用例

用户策略分配用例如表\ref{用户策略分配用例表}所示。用户策略分配是系统管理员给用户分配相应的访问策略,这里主要是针对存储系统账户进行策略分配,每个用户只能分配一个策略,该策略可以是系统内置的或自定义策略。
用户策略分配的方法是系统管理员选定某个策略后,所有的客户数据将在前端界面展现出来,可以同时选择多种客户数据,任何客户一旦选定可以与其他客户绑定的策略。
% 用户策略分配的方式是系统管理员选择一个策略时,所有的用户信息会在前端页面显示出来,可以同时选定多个用户信息,任意用户被选中即可为该用户绑定该策略。

\begin{center}
    \renewcommand\arraystretch{1.5}{
    \setlength{\tabcolsep}{8mm}{
	\begin{longtable}{|p{2.2cm}<{\centering}|p{9cm}|}
		\caption{用户策略分配用例表}\label{用户策略分配用例表}\\
        \hline
        \bf{用例编号} & UC10 \\
        \hline
        \bf{用例名称} & 用户策略分配用例 \\
		\hline
        \bf{用例参与者} & 系统管理员\\
		\hline
        \bf{前置条件} & 用户登录后进入策略管理系统后点击用户策略分配按钮 \\
		\hline
        \bf{触发条件} & 系统管理员选中指定策略后绑定用户 \\
		\hline
        \bf{后置条件} & 用户成功绑定策略 \\
		\hline
        \bf{正常流程} 
        & 1.系统管理员可以登录进入用户策略分配界面。\\
        & 2.系统管理员选中一个策略。\\
        & 3.系统管理员为该策略分配用户。\\
		\hline
        \bf{异常流程} 
        & 无\\
		\hline
	\end{longtable}}}
\end{center}

(11) 分组策略分配用例

分组策略分配用例如表\ref{分组策略分配用例表}所示。组策略分配是系统管理员给分组分配相应的访问策略,这里主要是针对存储系统账户的分组进行策略分配,每个分组只能分配一个策略,该策略可以是系统内置的或自定义策略。
分组策略分配的方式是系统管理员选择一个策略时,所有的分组信息会在前端页面显示出来,可以同时选定多个分组信息,任意分组被选中即可为该用户绑定该策略。

\begin{center}
    \renewcommand\arraystretch{1.5}{
    \setlength{\tabcolsep}{8mm}{
	\begin{longtable}{|p{2.2cm}<{\centering}|p{9cm}|}
		\caption{分组策略分配用例表}\label{分组策略分配用例表}\\
        \hline
        \bf{用例编号} & UC11 \\
        \hline
        \bf{用例名称} & 分组策略分配用例 \\
		\hline
        \bf{用例参与者} & 系统管理员\\
		\hline
        \bf{前置条件} & 用户登录后进入策略管理系统后点击分组策略分配按钮 \\
		\hline
        \bf{触发条件} & 系统管理员选中指定策略后绑定分组 \\
		\hline
        \bf{后置条件} & 分组成功绑定策略 \\
		\hline
        \bf{正常流程} 
        & 1.系统管理员可以进入到分组策略的界面。\\
        & 2.系统管理员选中一个策略。\\
        & 3.系统管理员为该策略分配分组。\\
		\hline
        \bf{异常流程} 
        & 无\\
		\hline
	\end{longtable}}}
\end{center}

(12) 角色权限分配用例

角色权限分配用例如表\ref{角色权限分配用例表}所示。
% 角色权限分配的应用案例类似于一般用户角色分配用例,角色权限分配的方法
% 是当超级管理员选定了某个角色时,所有的授权信息都会在前端界面展现出来,即可同时选择多种授权信息,而任何权限被选定者可以为
% 该角色绑定该授权。
% 角色权限分配用例类似于用户角色分配用例,角色权限分配的方式是超级管理员选择一个角色时,所有的权限信息会在前端页面显示
% 出来,可以同时选定多个权限信息,任意权限被选中即可为该角色绑定该权限。

\begin{center}
    \renewcommand\arraystretch{1.5}{
    \setlength{\tabcolsep}{8mm}{
	\begin{longtable}{|p{2.2cm}<{\centering}|p{9cm}|}
		\caption{角色权限分配用例表}\label{角色权限分配用例表}\\
        \hline
        \bf{用例编号} & UC12 \\
        \hline
        \bf{用例名称} & 角色权限分配用例 \\
		\hline
        \bf{用例参与者} & 超级管理员\\
		\hline
        \bf{前置条件} & 用户登录后进入权限管理系统后点击角色权限分配按钮 \\
		\hline
        \bf{触发条件} & 超级管理员选中指定角色后绑定权限 \\
		\hline
        \bf{后置条件} & 角色权限成功绑定 \\
		\hline
        \bf{正常流程} 
        & 1.超级管理员用户通过登陆并进入角色权限分配界面。\\
        & 2.超级管理员选中一个权限。\\
        & 3.超级管理员为该角色绑定权限。\\
		\hline
        \bf{异常流程} 
        & 无\\
		\hline
	\end{longtable}}}
\end{center}

(13) 网关路由管理用例

网关路由管理用例如表\ref{网关路由管理用例表}所示。
% 网关路由管理用例是指对API中网关模块的所有路径进行的相关管理,包括
% 建立路径、修改路由、删除路由和查看路由。
% API网关也是微服务架构中不可或缺的部分,微业务中每个的流量都以此为门户,因此对
% 各个模块所发送的每个要求的都经过了API网关这一唯一路径,那么要求就会经过API网关并经过路由项直接转发至整个的微业务。

% 网关路由管理用例是指对API网关模块的路由执行的相关操作,如创建路由、修改路由、
% 删除路由和查看路由。API网关是微服务架构中不可或缺的组件,微服务所有请求流量都以此作为入口,
% 向各个模块发出的所有请求的都经过API网关这一唯一通道,然后请求会被API网关根据路由项转发至各个微服务。

\begin{center}
    \renewcommand\arraystretch{1.5}{
    \setlength{\tabcolsep}{8mm}{
	\begin{longtable}{|p{2.2cm}<{\centering}|p{9cm}|}
		\caption{网关路由管理用例表}\label{网关路由管理用例表}\\
        \hline
        \bf{用例编号} & UC13 \\
        \hline
        \bf{用例名称} & 网关路由管理用例 \\
		\hline
        \bf{用例参与者} & 系统管理员\\
		\hline
        \bf{前置条件} & 用户在注册后进行授权管理,后单击网关路由管理按钮触发条件系统经营者可以对网关路由进行建立、撤销、更改和查询<br>后置条件系统经营者也对网关路由进行了建立、删除、修改和查看成功 \\
		\hline
        \bf{触发条件} & 系统管理员对网关路由进行创建、删除、修改和查看 \\
		\hline
        \bf{后置条件} & 系统管理员对网关路由进行创建、删除、修改和查看成功 \\
		\hline
        \bf{正常流程} 
        & 1.系统管理员在进入系统管理网站后,并选择系统路由。\\
        & 2.系统对用户身份进行验证, 确认为系统管理员后才可继续操作。\\
        & 3.系统对网关路由列表进行展示。\\
        & 4.系统管理员对网关路由执行创建、修改或删除操作。\\
		\hline
        \bf{异常流程} 
        & 无\\
         
		\hline
	\end{longtable}}}
\end{center}


(14) 服务器管理用例

服务器管理用例如表\ref{服务器管理用例表}所示。服务器管理用例主要是指超级管理员对服务器进行控制和修复,以及对服务器基本信息和服务器告警信息进行查看。
服务器控制主要包括服务器的启动和停止,服务器修复主要是根据服务器告警信息对简单问题进行快速修复,修复方式主要是运行针对常规问题的自动化修复脚本。
该管理权限属于最高级权限,只有超级管理员拥有。

\begin{center}
    \renewcommand\arraystretch{1.5}{
    \setlength{\tabcolsep}{8mm}{
	\begin{longtable}{|p{2.2cm}<{\centering}|p{9cm}|}
		\caption{服务器管理用例表}\label{服务器管理用例表}\\
        \hline
        \bf{用例编号} & UC14 \\
        \hline
        \bf{用例名称} & 服务器管理用例 \\
		\hline
        \bf{用例参与者} & 超级管理员\\
		\hline
        \bf{前置条件} & 用户先注册并成功进入运维系统,后单击服务器管理按钮 \\
		\hline
        \bf{触发条件} & 超级管理员对服务器进行控制、修复和信息查看 \\
		\hline
        \bf{后置条件} & 超级管理员对服务器进行控制、修复和信息查看成功 \\
		\hline
        \bf{正常流程} 
        & 1.超级用户登陆进入系统管理界面,并单击系统管理按钮。\\
        & 2.系统对用户身份信息进行了校验,确定是超级管理员后才可以继续运行。\\
        & 3.系统对服务器信息进行展示。\\
        & 4.系统管理员对服务器执行启停、修复、信息查看操作。\\
		\hline
        \bf{异常流程} 
        & 当前用户没有对服务器的管理权限,提示用户无权执行操作\\
         
		\hline
	\end{longtable}}}
\end{center}

(15) 集群管理用例

集群管理用例如表\ref{集群管理用例表}所示。服务器管理用例主要是指超级管理员对集群进行扩容、清理和信息查看。AliIO支持集群的动态扩容和清理,主要是通过相应的命令指定新的集群来扩展现有集群
或者直接清理现有集群,扩容的主要方式是采用水平扩容方式里的对等扩容和联邦扩容两种。该管理权限属于最高级权限,只有超级管理员拥有。

\begin{center}
    \renewcommand\arraystretch{1.5}{
    \setlength{\tabcolsep}{8mm}{
	\begin{longtable}{|p{2.2cm}<{\centering}|p{9cm}|}
		\caption{集群管理用例表}\label{集群管理用例表}\\
        \hline
        \bf{用例编号} & UC15 \\
        \hline
        \bf{用例名称} & 集群管理用例 \\
		\hline
        \bf{用例参与者} & 超级管理员\\
		\hline
        \bf{前置条件} & 用户在注册后进入运维系统后,单击集群管理按钮 \\
		\hline
        \bf{触发条件} & 超级管理员对集群进行扩容、清理和信息查看 \\
		\hline
        \bf{后置条件} & 超级管理员对集群进行扩容、清理和信息查看成功 \\
		\hline
        \bf{正常流程} 
        & 1.超级管理员可以登陆并进入系统的管理网站,然后选择集群管理系统。\\
        & 2.系统对用户身份信息进行了校验,确定是超级管理员后才可以继续运行。\\
        & 3.系统对集群列表进行展示。\\
        & 4.系统管理员对服务器执行扩容、清理、信息查看操作。\\
		\hline
        \bf{异常流程} 
        & 当前用户没有对集群的管理权限,提示用户无权执行操作\\
         
		\hline
	\end{longtable}}}
\end{center}

(16) 日志管理用例

日志管理用例如表\ref{日志管理用例表}所示。日志管理用例,主要负责平台日志和存储系统日志的查询与删除。管理平台日志主要包含用户登录管理系统所产生的相关记录,这些信息存在数据库内;
存储系统日志主要包含用户对Bucket、文件的操作等相关记录,这些信息会从存储服务器实时获取。

\begin{center}
    \renewcommand\arraystretch{1.5}{
    \setlength{\tabcolsep}{8mm}{
	\begin{longtable}{|p{2.2cm}<{\centering}|p{9cm}|}
		\caption{日志管理用例表}\label{日志管理用例表}\\
        \hline
        \bf{用例编号} & UC16 \\
        \hline
        \bf{用例名称} & 日志管理用例 \\
		\hline
        \bf{用例参与者} & 系统管理员\\
		\hline
        \bf{前置条件} & 用户先注册并成功登录运维系统,后单击日志管理按钮 \\
		\hline
        \bf{触发条件} & 系统管理员对日志进行查看和删除 \\
		\hline
        \bf{后置条件} & 系统管理员对日志进行查看和删除成功 \\
		\hline
        \bf{正常流程} 
        & 1.系统用户通过注册后加入到系统管理网站,并单击日志系统管理按键。\\
        & 2.先对系统身份进行校验,在确定是系统管理员后才可以继续运行。\\
        & 3.系统对日志列表进行展示。\\
        & 4.系统管理员对日志执行查看和删除操作。\\
		\hline
        \bf{异常流程} 
        & 当前用户没有对日志的管理权限,提示用户无权执行操作\\
         
		\hline
	\end{longtable}}}
\end{center}

(17) Bucket管理用例

Bucket管理用例如表\ref{Bucket管理用例表}所示。Bucket管理用例主要是指系统管理员和普通用户对Bucket执行一系列操作,如创建Bucket、删除Bucket、修改Bucket和查看Bucket,其中删除Bucket操作需要
在Bucket为空(即Bucket中的所有Object(文件)都已被删除)的前提下进行。

\begin{center}
    \renewcommand\arraystretch{1.5}{
    \setlength{\tabcolsep}{8mm}{
	\begin{longtable}{|p{2.2cm}<{\centering}|p{9cm}|}
		\caption{Bucket管理用例表}\label{Bucket管理用例表}\\
        \hline
        \bf{用例编号} & UC17 \\
        \hline
        \bf{用例名称} & Bucket管理用例 \\
		\hline
        \bf{用例参与者} & 系统管理员、普通用户\\
		\hline
        \bf{前置条件} & 系统用户在注册后进行文件管理,后单击Bucket管理按钮 \\
		\hline
        \bf{触发条件} & 系统管理员、普通用户对Bucket进行创建、删除、修改和查看 \\
		\hline
        \bf{后置条件} & 系统管理员、普通用户对Bucket进行创建、删除、修改和查看成功 \\
		\hline
        \bf{正常流程} 
        & 1.用户登录进入系统管理页面,并点击Bucket管理按钮。\\
        & 2.验证用户的访问策略和Bucket策略。\\
        & 3.系统对符合策略要求的Bucket列表进行展示。\\
        & 4.系统管理员和普通用户根据相应的访问策略对Bucket执行查看和删除操作。\\
		\hline
        \bf{异常流程} 
        & 1.从系统登录处进入系统管理界面,并单击Bucket管理按钮。\\
        & 2.当前用户的操作不符合其绑定的Bucket的访问策略,提示用户不可执行相关操作。\\
		\hline
	\end{longtable}}}
\end{center}

(18) 文件上传用例

文件上传用例如表\ref{文件上传用例表}所示。文件上传下达用例主要是用于普通用户上载自己的文件,可以实现各种各样的文件上传,还可以实现文件的批量上传。

\begin{center}
    \renewcommand\arraystretch{1.5}{
    \setlength{\tabcolsep}{8mm}{
	\begin{longtable}{|p{2.2cm}<{\centering}|p{9cm}|}
		\caption{文件上传用例表}\label{文件上传用例表}\\
        \hline
        \bf{用例编号} & UC18 \\
        \hline
        \bf{用例名称} & 文件上传用例 \\
		\hline
        \bf{用例参与者} & 普通用户\\
		\hline
        \bf{前置条件} & 用户本地存有需要上传的文件 \\
		\hline
        \bf{触发条件} & 用户登录后进入文件管理系统选择Bucket后点击文件上传按钮 \\
		\hline
        \bf{后置条件} & 浏览器将文件上传到后端存储系统 \\
		\hline
        \bf{正常流程} 
        & 1.用户可以登陆并进入系统的文件管理网站,选择Bucket并点击文件上传按钮。\\
        & 2.用户选择(可多选)需要上传的文件。\\
        & 3.验证用户的访问策略和Bucket策略,判断是否具有可写的策略。\\
        & 4.若有可写策略,则将文件上传到存储系统。\\ 
		\hline
        \bf{异常流程} 
        & 1.当前用户没有上传文件的可写策略, 系统提示用户无权上传。\\
        & 2.当前网络中断无法继续上传,系统提示用户上传中断。\\
		\hline
	\end{longtable}}}
\end{center}

(19) 文件下载用例

文件下载用例如表\ref{文件下载用例表}所示。文件下载用例主要是指普通用户从存储系统下载文件,可支持文件的批量下载。

\begin{center}
    \renewcommand\arraystretch{1.5}{
    \setlength{\tabcolsep}{8mm}{
	\begin{longtable}{|p{2.2cm}<{\centering}|p{9cm}|}
		\caption{文件下载用例表}\label{文件下载用例表}\\
        \hline
        \bf{用例编号} & UC19 \\
        \hline
        \bf{用例名称} & 文件下载用例 \\
		\hline
        \bf{用例参与者} & 普通用户\\
		\hline
        \bf{前置条件} & 进入某个Bucket的文件列表页面 \\
		\hline
        \bf{触发条件} & 用户点击下载按钮\\
		\hline
        \bf{后置条件} & 浏览器将文件从存储系统下载到本地 \\
		\hline
        \bf{正常流程} 
        & 1.直接登陆可以进入系统中的文件管理网站,选择Bucket并点击文件下载按钮。\\
        & 2.用户选择(可多选)需要下载的文件。\\
        & 3.验证用户的访问策略和Bucket策略,判断是否具有可读的策略。\\
        & 4.若有可读策略,则将文件下载到本地。\\ 
		\hline
        \bf{异常流程} 
        & 1.当前用户没有下载文件的可读策略, 系统提示用户无权下载。\\
        & 2.当前网络中断无法继续下载,系统提示用户下载中断。\\
		\hline
	\end{longtable}}}
\end{center}

\section{非功能性需求}
% 分布式对象存储管理系统的非功能性需求主要包含高性能、一致性、扩展性、简单性和安全性,非功能性需求是保证系统完成质量的基石,它在整个项目中拥有与功能性需求相同的价值。
% 要想使系统成为一个完善的、可持续迭代的系统,需要同时满足功能性和非功能性需求。
分布式对象存储管理系统的非功能性需求,主要包括了效率高、一致性、扩展、简单以及安全,非功能性要求是保证整个系统完成质量
的重要基础,其在整个项目中具有与功能性要求同等的价值。要想将系统变成一种完整的、可以不断迭代的体系,就必须同时满足功能
性与非功能性的要求。

\subsection{高性能}
在标准硬件上,AliIO具有高达183 GB / 秒 和 171 GB / 秒的读写速度,并且本系统使用的阿里自研存储新硬件AliFlash ZNS,
硬件读写性能相比传统的存储硬件又有大幅度提升,
因此系统整体拥有非常高的读写性能。同时,如果和一般的存储器系统一样,
将AliIO作为云原生应用程序的基础存储器,云原生应用程序将拥有更快的吞吐量和更少的延时。

\subsection{一致性}
% 本管理系统的底层的AliIO服务器通过组合各种实例,形成了统一的全局名称空间,通常情况下,一个分布式模式集最多可以由32个AliIO服务器组合而成,
% 并且多个分布式模式集可以组合成一个AliIO服务器联合,每个AliIO服务器联合都提供统一的管理员和名称空间,而且它们所能支持的分布式模式集合是无限数量的。
% 采用这种方式后,那些地理上分散的大型企业可以通过这个对象存储进行大规模扩展,
本系统的最底层的AliIO系统通过整合各种实例,可以建立起共同的全局名字空间,在一般情况下,一个分布式模式集最多能够由32
个AliIO系统组合而成,同时多个分布式模式集也能够组成一个AliIO服务器组合,因此各个AliIO服务器组合都可以拥有共同的管
理员和名字空间,同时它所能够使用的分布式模板集也是无限多的。采取这个方法后,一些地域上分散的大型公司能够利用这个对象
存储实现规模扩张,同时可以容纳S3 Select,MinSQL,Spark等各种应用程序。


\subsection{扩展性}
本系统底层的AliIO系统,单个集群能够直接与其他的AliIO集群结合到一起,从而产生全局名称空间,这些集群也能够按照需求穿越
许多不同类型的数据中心。创建的全域名称空间也能够利用添加更大的集群实现进一步扩充,直到达到目标规模。


\subsection{简单性}

本管理系统的文件上传和下载功能是通过简易的web页面并基于HTTP服务实现,操作非常简单方便,用户不用关心存储细节,减轻了用户的负担并增强了体验。
本系统底层的AliIO服务器的下载、部署和升级过程也非常简单,极大降低了使用和运维成本。


\subsection{安全性}

本系统保证数据安全,所有用户注册和登录过程都需要对信息和身份进行验证,操作时也会检查相应的权限。在底层的存储服务器上,服务器为每个用户和分组绑定了相应的访问策略,
用户只能按照相应的访问策略去访问文件,不能越权访问,实现了数据保密和数据隔离。而且AliIO支持身份管理中最先进的标准,它摒弃了在配置文件和数据库中存储密码的方式,
这样不仅保证了集中访问,还保证了密码是临时且轮换的。

\section{本章小结}

本文先是对分布式数据存储处理技术进行了回顾,然后针对系统的功能性要求进行了研究,并基于系统的功能性要求进行了系统的例表,以表格的方式详细介绍了各个需求的用例,
最后分别从高性能、一致性、扩展性、简单性和安全性五个方面介绍了本系统的非功能性需求。
这章对分布式对象存储管理系统软件的技术需求做出了具体的介绍,为后面的文章打下了基础。
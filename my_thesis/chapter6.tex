\renewcommand{\arraystretch}{1.5}

\chapter{分布式对象存储系统管控平台系统测试与分析}

系统测试是系统开发过程不可或缺的重要步骤,是项目质量保障的关键点。本章首先简要介绍了测试环境和测试工具,然后对第三章中讨论到的功能性需求和非功能性需求进行了试验,
并收集和分析了测试成果,以判断系统功能是否符合预期需求。

\section{系统测试环境}

在系统测试开始前,必须提供良好的系统测试环境,系统测试环境的准备和配置往往决定着测试结果的准确性,为了保证分布式对象存储系统管控平台项目的功能性和非功能性测试的结果能准确全面,
在此详细介绍本系统在进行测试时所使用的测试环境和工具,如表\ref{测试环境表}所示。

\begin{center}
    \renewcommand\arraystretch{1.5}{
    \setlength{\tabcolsep}{5mm}{
	\begin{longtable}{|p{2.2cm}<{\centering}|p{3.4cm}<{\centering}|p{5.6cm}<{\centering}|}
		\caption{测试环境表}\label{测试环境表}\\
		\hline
        \bf{测试环境} & \bf{说明} & \bf{内容} \\
        \hline
        \multirow{4}*{测试硬件} & 电脑 & 联想电脑\\
        \cline{2-3}
        & 操作系统 & Windows 11 22H2 版本\\
        \cline{2-3}
         & 处理器 & AMD Ryzen 7 4800U\\
        \cline{2-3}
         & 内存 & 16GB\\
        \hline
        \multirow{6}*{测试软件} & 浏览器 & Google Chrome\\
        \cline{2-3}
        & 数据库 & MySQL\\
        \cline{2-3}
        & 黑盒测试 & Selenium\\
        \cline{2-3}
        & 单元测试 & Vitest\\
        \cline{2-3}
        & 接口测试 & Postman\\
        \cline{2-3}
        & 压力测试 & Jmeter\\
        \hline
	\end{longtable}}}
    \vspace{-1cm}
\end{center}

在测试硬件上,本平台的测试电脑使用联想电脑小新 Pro 13,它的操作系统是Windows 11 22H2 版本,处理器是AMD Ryzen 7 4800U,内存大小是16GB。经过评估,该测试硬件的配置满足本平
台运行的基本要求,可以保证系统稳定可靠的运行。

在测试软件和工具上,使用Google Chrome浏览器和MySQL数据库作为整个系统运行的基础组件,为了全方位的测试系统的功能与性能,利用Selenium进行黑盒测试\cite{kng222eji},Vitest进行单元测试\cite{kng2562eji},
Postman进行接口测试,以及Jmeter进行压力测试\cite{kng232eji}。这些测试软件和工具可以对系统的各方面进行测试,达到预期的测试效果。

此外,由于本平台是分布式对象存储系统管控平台,它必须依托于对象存储系统AliIO运行,故此处也介绍一下存储系统的环境配置,环境信息如表\ref{存储系统环境表}所示,环境实体如图\ref{fig:存储系统环境实体图}所示。

\begin{center}
    \renewcommand\arraystretch{1.5}{
    \setlength{\tabcolsep}{5mm}{
	\begin{longtable}{|p{2.2cm}<{\centering}|p{3.4cm}<{\centering}|p{5.6cm}<{\centering}|}
		\caption{存储系统环境表}\label{存储系统环境表}\\
		\hline
        \bf{节点} & \bf{IP} & \bf{服务器机型\&操作系统} \\
        \hline
        AliIO-Node1 & 192.168.201.200 & \multirow{4}*{磐久服务器\&Alios7}\\
        \cline{1-2}
        AliIO-Node2 & 192.168.201.157 & \\
        \cline{1-2}
        AliIO-Node3 & 192.168.201.54 & \\
        \cline{1-2}
        AliIO-Node4 & 192.168.201.210 & \\
        \hline
	\end{longtable}}}
    \vspace{-1cm}
\end{center}


\begin{figure}[htb]
    \centering
    \includegraphics[width=1\textwidth]{my_figures/chapter6/环境实体图.jpg}
    \caption{存储系统环境实体图}
    \label{fig:存储系统环境实体图}
%     \note{注:图注的内容不宜放到图题中。}
\end{figure}


\section{功能性测试}

本部分分别测试系统的各个功能模块,按照需求分析中的功能需求对各模块的功能点设计测试用例,按照测试步骤在系统中执行测试,看测试结果是否符合测试预期,如果符合,则说明该功能点
无误,如果不符合,这说明该功能点存在问题,需要进一步修复。如果一个模块的所有功能测试点均测试无误,则说明该模块的功能通过了测试,可以安全运行。


\subsection{注册认证模块功能测试}

注册认证模块的测试用例表如表\ref{tab:注册认证模块测试用例表}所示,该用例描述了系统中的注册认证模块相关功能的几个测试点,包括用户正常注册、信息格式校验、信息重复性校验、
重复注册校验、正常登录、异常登录和身份验证。

\begin{longtable}{|m{0.16\linewidth}|m{0.3\linewidth}|m{0.3\linewidth}|m{0.11\linewidth}|}

    \caption{注册认证模块测试用例表}\label{tab:注册认证模块测试用例表} \\
     \endfirsthead
     \multicolumn{4}{c}{ \bf{表 \thetable\ 注册认证模块测试用例表(续表)} } \\
     \hline
     测试点   & 测试步骤                          & 测试预期             & 测试结果 \\
     \hline
     \endhead
     \hline
     用例编号  & \multicolumn{3}{l|}{TC01} \\
     \hline
     测试类型  & \multicolumn{3}{l|}{功能性测试}                                 \\
     \hline
     测试目的  & \multicolumn{3}{l|}{测试注册认证模块的功能是否正常}                          \\
     \hline
     对应需求编号 & \multicolumn{3}{l|}{UC01、UC02} \\ \hline
     前提条件  & \multicolumn{3}{l|}{用户已成功运行系统进入注册或登录页面}                        \\
     \hline
     期望结果  & \multicolumn{3}{l|}{用户可以正常注册登录系统}                           \\
     \hline
     实际结果  & \multicolumn{3}{l|}{测试成功}                                 \\
     \hline
     测试点   & 测试步骤                          & 测试预期             & 测试结果 \\
     \hline
     用户正常注册 & \parbox[t]{4.5cm}{\hangindent=1em\hangafter=1 1.在注册页面输入合法的用户名、密码等信息} \vspace{-0.8mm} \newline 2.点击注册按钮 & 用户注册成功   & 测试通过 \\
     \hline
     信息格式校验 & \parbox[t]{4.5cm}{\hangindent=1em\hangafter=1 1.在注册页面输入格式错误的用户名、密码等信息} \vspace{-0.8mm} \newline{2.点击注册按钮} & 显示用户名或密码非法   & 测试通过 \\
     \hline
     信息重复性校验 & \parbox[t]{4.5cm}{\hangindent=1em\hangafter=1 1.在注册页面输入已经存在的用户名或邮箱} \vspace{-0.8mm} \newline{2.点击注册按钮} & 显示用户名或邮箱不可用   & 测试通过 \\
     \hline
     重复注册校验 & \parbox[t]{4.5cm}{\hangindent=1em\hangafter=1 1.在注册页面输入已经存在的用户名和密码} \vspace{-0.8mm} \newline{2.点击注册按钮} & 显示该用户已注册   & 测试通过 \\
     \hline
     验证码校验 & \parbox[t]{4.5cm}{\hangindent=1em\hangafter=1 1.在注册页面输入不正确的验证码} \vspace{-0.8mm} \newline{2.点击注册按钮} & 显示验证码错误并刷新验证码   & 测试通过 \\
     \hline
     正常登录 & \parbox[t]{4.5cm}{\hangindent=1em\hangafter=1 1.在登录页面输入已注册成功的用户名、密码信息} \vspace{-0.8mm} \newline{2.点击登录按钮}  & 用户登录成功 & 测试通过 \\
     \hline
     异常登录 & \parbox[t]{4.5cm}{\hangindent=1em\hangafter=1 1.在登录页面输入不正确的用户名、密码信息} \vspace{-0.8mm} \newline{2.点击登录按钮}  & 用户登录失败 & 测试通过 \\
     \hline
     身份验证 & \parbox[t]{4.5cm}{\hangindent=1em\hangafter=1 1.用户登录成功后,先短时间内点击其他页面} \vspace{-0.8mm} \newline \parbox[t]{4.5cm}{\hangindent=1em\hangafter=1 2.隔很长一段时间再点击页面} \vspace{-0.8mm} & 刚开始可以正常点击其他页面,之后则跳转到登录页面 & 测试通过 \\
     \hline  
\end{longtable}

\begin{figure}[htb]
    \centering
    \includegraphics[width=1\textwidth]{my_figures/chapter6/登录失败.png}
    \caption{登录失败}
    \label{fig:登录失败}
%     \note{注:图注的内容不宜放到图题中。}
\end{figure}




\begin{figure}[htb]
    \centering
    \includegraphics[width=1\textwidth]{my_figures/chapter6/登录成功.png}
    \caption{登录成功}
    \label{fig:登录成功}
%     \note{注:图注的内容不宜放到图题中。}
\end{figure}

如图\ref{fig:登录失败}所示,进入注册登录页面后,如果输入错误的用户名或密码,将会提示登录失败并需要重新输入。如果是新用户,则需要完成注册流程。在注册过程中,您可以选择普通用户或管理员身份。注册成功后,您只需要输入已经注册的用户名和密码即可成功登录,并跳转到系统首页,如图\ref{fig:登录成功}所示。


% 根据测试结果,注册认证模块各项功能正常,能够满足用户在生产环境中注册、登录和身份验证方面的需求。

\subsection{权限控制模块功能测试}

权限控制模块的测试用例表如表\ref{tab:权限控制模块测试用例表}所示,该用例描述了系统中的权限控制模块相关功能的几个测试点,包括增加权限、删除权限、修改权限、查看权限、查看角色、
分配角色和用户权限不足。

\begin{longtable}{|m{0.16\linewidth}|m{0.3\linewidth}|m{0.3\linewidth}|m{0.11\linewidth}|}

    \caption{权限控制模块测试用例表}\label{tab:权限控制模块测试用例表} \\
     \endfirsthead
     \multicolumn{4}{r}{ \bf{续表} } \\
    %  \hline
    %  测试点   & 测试步骤                          & 测试预期             & 测试结果 \\
     \hline
     \endhead
     \hline
     用例编号  & \multicolumn{3}{l|}{TC02} \\
     \hline
     测试类型  & \multicolumn{3}{l|}{功能性测试}                                 \\
     \hline
     测试目的  & \multicolumn{3}{l|}{测试权限控制模块的功能是否正常}                          \\
     \hline
     对应需求编号 & \multicolumn{3}{l|}{UC06、UC07} \\ \hline
     前提条件  & \multicolumn{3}{l|}{系统管理员已成功登录系统}                        \\
     \hline
     期望结果  & \multicolumn{3}{l|}{系统管理员可以正常使用权限控制模块的各项功能}                           \\
     \hline
     实际结果  & \multicolumn{3}{l|}{测试成功}                                 \\
     \hline
     测试点   & 测试步骤                          & 测试预期             & 测试结果 \\
     \hline
     增加权限 & 1.点击创建权限按钮 \newline{2.输入权限信息} \newline{3.点击提交按钮} \newline \parbox[t]{4.5cm}{\hangindent=1em\hangafter=1 4.重新查看权限信息是否增加} \vspace{-0.8mm} & 权限创建成功   & 测试通过 \\
     \hline
     删除权限 & 1.点击删除权限按钮 \newline{2.点击取消删除按钮} \newline{3.再次点击权限删除按钮} \newline{4.点击确认删除按钮}& 权限删除成功 & 测试通过 \\
     \hline
     修改权限 & 1.点击修改权限按钮 \newline{2.填入新的权限信息} \newline{3.点击提交按钮} \newline \parbox[t]{4.5cm}{\hangindent=1em\hangafter=1 4.重新查看权限信息是否更新} \vspace{-0.8mm} & 权限修改成功 & 测试通过 \\
     \hline
     查看权限 & 1.点击查看权限按钮 \newline{2.选择不同的角色}   & 查看到对应角色下的所有权限 & 测试通过 \\
     \hline
     查看角色 & 1.点击查看角色按钮  & 查看到所有角色信息 & 测试通过 \\
     \hline
     分配角色 & 1.点击分配角色按钮 \newline \parbox[t]{4.5cm}{\hangindent=1em\hangafter=1 2.选择一个权限或用户,点击确认}  \newline \parbox[t]{4.5cm}{\hangindent=1em\hangafter=1 3.选择多个权限或用户,点击确认} \vspace{-0.8mm} & 查看角色与权限和用户的绑定信息 & 测试通过 \\
     \hline
     用户权限不足 & 1.用普通用户登录系统 \newline{2.点击系统监控页面}  \newline{3.点击文件存取页面}& 用户进入系统监控页面失败,进入文件存取页面成功 & 测试通过 \\
     \hline 
\end{longtable}


\begin{figure}[htb]
    \centering
    \includegraphics[width=1\textwidth]{my_figures/chapter6/角色信息.png}
    \caption{角色查询}
    \label{fig:角色信息}
%     \note{注:图注的内容不宜放到图题中。}
\end{figure}


如图\ref{fig:角色信息}所示,进入角色查询页面后,可以查询到企业管理员的角色信息,显示角色名称、创建时间、角色权限等信息,企业管理员角色属于系统默认角色,所以不能对
其进行删除操作,只能进行查询。



% 根据测试结果,权限控制模块各项功能正常,能够满足系统管理员在生产环境中对权限和角色的管理需求。

\subsection{策略控制模块功能测试}

策略控制模块的测试用例表如表\ref{tab:策略控制模块测试用例表}所示,该用例描述了系统中的策略控制模块相关功能的几个测试点,包括创建用户、删除用户、修改用户、查看用户、禁用用户、
创建访问策略、删除访问策略、查看访问策略、修改访问策略、禁用访问策略和分配访问策略。

\begin{longtable}{|m{0.16\linewidth}|m{0.3\linewidth}|m{0.3\linewidth}|m{0.11\linewidth}|}

    \caption{策略控制模块测试用例表}\label{tab:策略控制模块测试用例表} \\
     \endfirsthead
     \multicolumn{4}{r}{ \bf{续表} } \\
     \hline
     测试点   & 测试步骤                          & 测试预期             & 测试结果 \\
     \hline
     \endhead
     \hline
     用例编号  & \multicolumn{3}{l|}{TC03} \\
     \hline
     测试类型  & \multicolumn{3}{l|}{功能性测试}                                 \\
     \hline
     测试目的  & \multicolumn{3}{l|}{测试策略控制模块的功能是否正常}                          \\
     \hline
     对应需求编号 & \multicolumn{3}{l|}{UC03、UC04、UC05} \\ \hline
     前提条件  & \multicolumn{3}{l|}{管理员已成功登录系统}                        \\
     \hline
     期望结果  & \multicolumn{3}{l|}{管理员可以正常使用策略控制模块的各项功能}                           \\
     \hline
     实际结果  & \multicolumn{3}{l|}{测试成功}                                 \\
     \hline
     测试点   & 测试步骤                          & 测试预期             & 测试结果 \\
     \hline
     创建用户 & \parbox[t]{4.5cm}{\hangindent=1em\hangafter=1 1.管理员在用户管理页面点击新增用户}\newline{2.输入用户信息} \newline{3.点击提交按钮} \newline \parbox[t]{4.5cm}{\hangindent=1em\hangafter=1 4.查看新用户信息是否增加}& 用户创建成功   & 测试通过 \\
     \hline
     删除用户 & \parbox[t]{4.5cm}{\hangindent=1em\hangafter=1 1.管理员在用户管理页面选择用户并点击删除} \vspace{-0.8mm} & 页面提示用户删除成功 & 测试通过 \\
     \hline
     修改用户 & \parbox[t]{4.5cm}{\hangindent=1em\hangafter=1 1.管理员在用户管理页面选择用户并点击修改} \vspace{-0.8mm} \newline{2.输入新的用户信息} \newline{3.点击提交按钮} \newline \parbox[t]{4.5cm}{\hangindent=1em\hangafter=1 4.刷新页面,重新查看用户信息是否更新} \vspace{-0.8mm} & 用户修改成功 & 测试通过 \\
     \hline
     查看用户 & \parbox[t]{4.5cm}{\hangindent=1em\hangafter=1 1.管理员在用户管理页面选择用户并点击查看} \vspace{-0.8mm} & 查询到用户信息 & 测试通过 \\
     \hline
     禁用用户 & \parbox[t]{4.5cm}{\hangindent=1em\hangafter=1 1.管理员在用户管理页面选择用户并点击查看} \vspace{-0.8mm} & 查询到用户信息 & 测试通过 \\
     \hline
     创建访问策略 & \parbox[t]{4.5cm}{\hangindent=1em\hangafter=1 1.管理员在策略管理页面点击新增策略} \vspace{-0.8mm} \newline{2.输入策略信息} \newline{3.点击提交按钮} \newline \parbox[t]{4.5cm}{\hangindent=1em\hangafter=1 4.查看新策略信息是否增加}  & 策略新增成功 & 测试通过 \\
     \hline
     删除访问策略 & \parbox[t]{4.5cm}{\hangindent=1em\hangafter=1 1.管理员在策略管理页面选择一个或多个策略点击删除} \vspace{-0.8mm} \newline{2.查看策略列表}  & 策略删除成功 & 测试通过 \\
     \hline
     修改访问策略 & \parbox[t]{4.5cm}{\hangindent=1em\hangafter=1 1.管理员在策略管理页面选择一个策略点击修改} \vspace{-0.8mm} \newline \parbox[t]{4.5cm}{\hangindent=1em\hangafter=1 2.输入新的策略信息并点击提交} \vspace{-0.8mm} \newline{3.重新查看策略信息}& 策略修改成功 & 测试通过 \\
     \hline
     查看访问策略 & \parbox[t]{4.5cm}{\hangindent=1em\hangafter=1 1.管理员在策略管理页面选择一个策略点击查看} \vspace{-0.8mm} & 查看到完整的策略信息 & 测试通过 \\
     \hline
     禁用访问策略 & \parbox[t]{4.5cm}{\hangindent=1em\hangafter=1 1.管理员在策略管理页面选择一个策略点击禁用} \vspace{-0.8mm}  & 策略无法被分配 & 测试通过 \\
     \hline
     分配访问策略 & \parbox[t]{4.5cm}{\hangindent=1em\hangafter=1 1.管理员在用户管理页面选择一个用户点击策略分配} \vspace{-0.8mm} \newline{2.在策略菜单里选择策略} \newline \parbox[t]{4.5cm}{\hangindent=1em\hangafter=1 3.登录分配策略的用户进行文件存取操作} \vspace{-0.8mm} & 用户策略分配成功 & 测试通过 \\
     \hline 
\end{longtable}

% \begin{figure}[htb]
%     \centering
%     \includegraphics[width=1\textwidth]{my_figures/chapter6/策略创建.png}
%     \caption{策略创建}
%     \label{fig:策略创建}
% %     \note{注:图注的内容不宜放到图题中。}
% \end{figure}


\begin{figure}[htb]
    \centering
    \includegraphics[width=1\textwidth]{my_figures/chapter6/策略创建成功.png}
    \caption{策略创建成功}
    \label{fig:策略创建成功}
%     \note{注:图注的内容不宜放到图题中。}
\end{figure}

如图\ref{fig:策略创建成功}所示,进入策略创建页面后,需要填写策略名称和策略内容,然后根据需要选择策略版本和策略的操作列表。最后,需要确定策略的类型,即是IAM权限策略
还是基于资源策略。此外,如果选择了不启用策略选项,则策略将被创建,但不能被使用。策略成功创建后,系统会发送提示消息。



% 根据测试结果,策略控制模块各项功能正常,能够满足管理员在生产环境中进行用户管理、策略管理和分配方面的需求。

\subsection{文件存取模块功能测试}

文件存取模块的测试用例表如表\ref{tab:文件存取模块测试用例表}所示,该用例描述了系统中的文件存取模块相关功能的几个测试点,包括创建桶、删除桶、修改桶名称、禁用桶、修改桶的访问策略、
修改桶的生命周期、普通文件上传、视频文件上传、多文件混合上传、普通文件下载、视频文件下载和多文件混合下载。

\begin{longtable}{|m{0.16\linewidth}|m{0.3\linewidth}|m{0.3\linewidth}|m{0.11\linewidth}|}

    \caption{文件存取模块测试用例表}\label{tab:文件存取模块测试用例表} \\
     \endfirsthead
     \multicolumn{4}{r}{ \bf{续表} } \\
     \hline
     测试点   & 测试步骤                          & 测试预期            & 测试结果 \\
     \hline
     \endhead
     \hline
     用例编号  & \multicolumn{3}{l|}{TC04} \\
     \hline
     测试类型  & \multicolumn{3}{l|}{功能性测试}                                 \\
     \hline
     测试目的  & \multicolumn{3}{l|}{测试文件存取模块的功能是否正常}                          \\
     \hline
     对应需求编号 & \multicolumn{3}{l|}{UC08、UC09} \\ \hline
     前提条件  & \multicolumn{3}{l|}{用户已成功登录系统}                        \\
     \hline
     期望结果  & \multicolumn{3}{l|}{用户可以正常使用文件存取模块的各项功能}                           \\
     \hline
     实际结果  & \multicolumn{3}{l|}{测试成功}                                 \\
     \hline
     测试点   & 测试步骤                          & 测试预期             & 测试结果 \\
     \hline
     创建桶 & \parbox[t]{4.5cm}{\hangindent=1em\hangafter=1 1.用户在桶管理页面点击创建桶} \vspace{-0.8mm}\newline{2.填入桶名称、策略等信息} \newline{3.点击提交按钮} \newline \parbox[t]{4.5cm}{\hangindent=1em\hangafter=1 4.重新查看桶信息是否增加}& 权限创建成功   & 测试通过 \\
     \hline
     删除桶 & \parbox[t]{4.5cm}{\hangindent=1em\hangafter=1 1.用户在桶管理页面选择桶点击删除} \vspace{-0.8mm} \newline{2.点击确认删除全部文件} & 删除桶成功 & 测试通过 \\
     \hline
     修改桶名称 & \parbox[t]{4.5cm}{\hangindent=1em\hangafter=1 1.用户在桶管理页面选择桶名称点击重命名} \vspace{-0.8mm} \newline{2.输入新的桶名称} \newline{3.点击确认按钮} \newline{4.重新查看桶名称} & 桶名称修改成功 & 测试通过 \\
     \hline
     禁用桶 & \parbox[t]{4.5cm}{\hangindent=1em\hangafter=1 1.用户在桶管理页面选择桶点击禁用} \vspace{-0.8mm} \newline \parbox[t]{4.5cm}{\hangindent=1em\hangafter=1 2.用户重新登录对桶进行选取} \vspace{-0.8mm}  & 桶无法被选取和操作 & 测试通过 \\
     \hline
     修改桶的访问策略 & \parbox[t]{4.5cm}{\hangindent=1em\hangafter=1 1.用户在桶管理页面选择桶点击策略选择} \vspace{-0.8mm} \newline{2.更换访问策略}   & 桶的访问策略修改 成功 & 测试通过 \\
     \hline
     修改桶的生命周期 & \parbox[t]{4.5cm}{\hangindent=1em\hangafter=1 1.用户在桶管理页面选择桶点击修改生命周期} \vspace{-0.8mm} \newline{2.输入新的生命周期} & 桶的生命周期修改成功 & 测试通过 \\
     \hline
     普通文件上传 & \parbox[t]{4.5cm}{\hangindent=1em\hangafter=1 1.用户在文件存取页面点击文件上传} \vspace{-0.8mm} \newline \parbox[t]{4.5cm}{\hangindent=1em\hangafter=1 2.选择本地普通文件,并选择存储桶,点击确认} \vspace{-0.8mm} \newline{3.查看文件列表}& 文件上传成功 & 测试通过 \\
     \hline
     视频文件上传 & \parbox[t]{4.5cm}{\hangindent=1em\hangafter=1 1.用户在文件存取页面点击文件上传} \vspace{-0.8mm} \newline \parbox[t]{4.5cm}{\hangindent=1em\hangafter=1 2.选择本地视频文件,并选择存储桶,点击确认} \vspace{-0.8mm} \newline{3.查看文件列表}& 文件上传成功 & 测试通过 \\
     \hline
     多文件混合上传 & \parbox[t]{4.5cm}{\hangindent=1em\hangafter=1 1.用户在文件存取页面点击文件上传} \vspace{-0.8mm} \newline \parbox[t]{4.5cm}{\hangindent=1em\hangafter=1 2.同时选择多个文件,包括普通文件和视频文件} \vspace{-0.8mm} \newline{3.查看文件列表}& 文件上传成功 & 测试通过 \\
     \hline
     普通文件下载 & \parbox[t]{4.5cm}{\hangindent=1em\hangafter=1 1.用户在文件存取页面点击文件下载} \vspace{-0.8mm} \newline \parbox[t]{4.5cm}{\hangindent=1em\hangafter=1 2.选择存储桶和对应的普通文件} \vspace{-0.8mm} \newline{3.选择下载到本地的位置} \newline{4.查看本地文件}& 文件下载成功 & 测试通过 \\
     \hline
     视频文件下载 &\parbox[t]{4.5cm}{\hangindent=1em\hangafter=1 1.用户在文件存取页面点击文件下载} \vspace{-0.8mm} \newline \parbox[t]{4.5cm}{\hangindent=1em\hangafter=1 2.选择存储桶和对应的视频文件} \vspace{-0.8mm} \newline{3.选择下载到本地的位置} \newline{4.查看本地文件}& 文件下载成功 & 测试通过 \\
     \hline
     多文件混合下载 & \parbox[t]{4.5cm}{\hangindent=1em\hangafter=1 1.用户在文件存取页面点击文件下载} \vspace{-0.8mm} \newline \parbox[t]{4.5cm}{\hangindent=1em\hangafter=1 2.同时选择多个存储桶中不同的文件} \vspace{-0.8mm} \newline{3.选择下载到本地的位置}& 文件下载成功 & 测试通过 \\
     \hline
    
\end{longtable}

\begin{figure}[htb]
    \centering
    \includegraphics[width=1\textwidth]{my_figures/chapter6/存储桶创建成功.png}
    \caption{存储桶创建成功}
    \label{fig:存储桶创建成功}
%     \note{注:图注的内容不宜放到图题中。}
\end{figure}

如图\ref{fig:存储桶创建成功}所示,进入存储桶创建页面后,需填写存储桶的名称、对象数限制值、存储桶的生命周期、访问策略,以及是否启用对象锁定和版本控制等功能选项。
此外,存储桶的生命周期默认单位为天。在填写完所有信息后,点击页面右下方的“创建”按钮,系统会检查填写信息并向您显示存储桶创建成功的提示信息,方便进行下一步操作。


\begin{figure}[htb]
    \centering
    \includegraphics[width=1\textwidth]{my_figures/chapter6/文件下载成功.png}
    \caption{文件下载成功}
    \label{fig:文件下载成功}
%     \note{注:图注的内容不宜放到图题中。}
\end{figure}

如图\ref{fig:文件下载成功}所示,进入文件下载页面后,可以看到每个文件的名称、修改时间和大小,并发现每个文件的最右侧都有下载和删除按钮。如果需要下载某个文件,只
需直接点击其右侧的下载按钮,文件将开始下载。下载成功后,系统将向您展示文件下载完成的提示信息。如果需要同时下载多个文件,可点击页面右下角的“批量下载”按钮,然后选
择多个文件进行下载。

\begin{figure}[htb]
    \centering
    \includegraphics[width=1\textwidth]{my_figures/chapter6/文件上传.png}
    \caption{文件上传}
    \label{fig:文件上传}
%     \note{注:图注的内容不宜放到图题中。}
\end{figure}

如图\ref{fig:文件上传}所示,进入文件上传页面后,您可以看到已经上传文件的名称、上传时间和状态。文件上传成功后,状态将显示为“success”,如果上传失败,则会显示“fail”。
如果需要同时上传多个文件,可点击页面右下角的“批量上传”按钮,然后选择多个文件进行上传。

% 根据测试结果,文件存取模块各项功能正常,能够满足用户在生产环境中进行桶管理、文件上传和下载方面的需求。

\subsection{系统监控模块功能测试}

系统监控模块的测试用例表如表\ref{tab:系统监控模块测试用例表}所示,该用例描述了系统中的系统监控模块相关功能的几个测试点,包括查看存储桶数目、查看文件数目、查看已用容量、
查看可用容量、查看连接状态、展示状态监测图、查看事件列表、查看单个事件信息、删除单个事件信息、删除全部事件信息、查看事件分析报告、查看告警通知和查看处理结果。

\begin{longtable}{|m{0.16\linewidth}|m{0.3\linewidth}|m{0.3\linewidth}|m{0.11\linewidth}|}

    \caption{系统监控模块测试用例表}\label{tab:系统监控模块测试用例表} \\
     \endfirsthead
     \multicolumn{4}{r}{ \bf{续表} } \\
     \hline
     测试点   & 测试步骤                          & 测试预期             & 测试结果 \\
     \hline
     \endhead
     \hline
     用例编号  & \multicolumn{3}{l|}{TC05} \\
     \hline
     测试类型  & \multicolumn{3}{l|}{功能性测试}                                 \\
     \hline
     测试目的  & \multicolumn{3}{l|}{测试系统监控模块功能是否正常}                          \\
     \hline
     对应需求编号 & \multicolumn{3}{l|}{UC10} \\ \hline
     前提条件  & \multicolumn{3}{l|}{系统管理员已成功登录系统}                        \\
     \hline
     期望结果  & \multicolumn{3}{l|}{系统管理员可以使用系统监控模块的各项功能}                           \\
     \hline
     实际结果  & \multicolumn{3}{l|}{测试成功}                                 \\
     \hline
     测试点   & 测试步骤                          & 测试预期             & 测试结果 \\
     \hline
     查看存储桶数目 & \parbox[t]{4.5cm}{\hangindent=1em\hangafter=1 1.用户在状态监测页面点击查看桶数目} \vspace{-0.8mm} & 页面显示桶的数目   & 测试通过 \\
     \hline
     查看文件数目 & \parbox[t]{4.5cm}{\hangindent=1em\hangafter=1 1.用户在状态监测页面点击查看文件数目} \vspace{-0.8mm} & 页面显示文件的数目 & 测试通过 \\
     \hline
     查看已用容量 & \parbox[t]{4.5cm}{\hangindent=1em\hangafter=1 1.用户在状态监测页面点击查看已用容量} \vspace{-0.8mm} & 页面显示已用容量 & 测试通过 \\
     \hline
     查看可用容量 & \parbox[t]{4.5cm}{\hangindent=1em\hangafter=1 1.用户在状态监测页面点击查看可用容量} \vspace{-0.8mm} & 页面显示可用容量 & 测试通过 \\
     \hline
     查看连接状态 & \parbox[t]{4.5cm}{\hangindent=1em\hangafter=1 1.用户在状态监测页面点击查看连接状态} \vspace{-0.8mm} & 页面显示连接状态 & 测试通过 \\
     \hline
     展示状态监测图 & \parbox[t]{4.5cm}{\hangindent=1em\hangafter=1 1.用户在状态监测页面点击状态监测图} \vspace{-0.8mm} & 页面显示状态监测图 & 测试通过 \\
     \hline
     查看事件列表 & \parbox[t]{4.5cm}{\hangindent=1em\hangafter=1 1.用户在事件跟踪页面点击查看事件列表} \vspace{-0.8mm} & 页面显示事件列表 & 测试通过 \\
     \hline
     查看单个事件信息 & \parbox[t]{4.5cm}{\hangindent=1em\hangafter=1 1.用户在事件跟踪页面点击查看事件列表} \vspace{-0.8mm} \newline \parbox[t]{4.5cm}{\hangindent=1em\hangafter=1 2.选择单个事件信息点击查看} \vspace{-0.8mm} & 页面显示单个事件信息 & 测试通过 \\
     \hline
     删除单个事件信息 & \parbox[t]{4.5cm}{\hangindent=1em\hangafter=1 1.用户在事件跟踪页面点击查看事件列表} \vspace{-0.8mm} \newline \parbox[t]{4.5cm}{\hangindent=1em\hangafter=1 2.选择单个事件信息点击删除} \vspace{-0.8mm} & 页面显示事件删除成功 & 测试通过 \\
     \hline
     删除全部事件信息 & \parbox[t]{4.5cm}{\hangindent=1em\hangafter=1 1.用户在事件跟踪页面点击查看事件列表} \vspace{-0.8mm} \newline \parbox[t]{4.5cm}{\hangindent=1em\hangafter=1 2.选择所有事件信息点击删除} \vspace{-0.8mm} & 页面显示所有事件全部删除 & 测试通过 \\
     \hline
     查看事件分析报告 & \parbox[t]{4.5cm}{\hangindent=1em\hangafter=1 1.用户在事件跟踪页面点击生成分析报告} \vspace{-0.8mm} \newline \parbox[t]{4.5cm}{\hangindent=1em\hangafter=1 2.打开生成的分析报告} & 分析报告打开成功 & 测试通过 \\
     \hline
     查看告警通知 & \parbox[t]{4.5cm}{\hangindent=1em\hangafter=1 1.用户在告警通知页面点击查看通知} \vspace{-0.8mm} & 页面显示告警通知 & 测试通过 \\
     \hline
    %  查看处理结果 & \parbox[t]{4.5cm}{\hangindent=1em\hangafter=1 1.用户在告警通知页面点击查看处理结果} \vspace{-0.8mm} & 页面显示处理结果 & 测试通过 \\
    %  \hline
\end{longtable}

\begin{figure}[htb]
    \centering
    \includegraphics[width=1\textwidth]{my_figures/chapter6/状态监测.png}
    \caption{状态监测}
    \label{fig:状态监测}
%     \note{注:图注的内容不宜放到图题中。}
\end{figure}

如图\ref{fig:状态监测}所示,进入系统监控的状态监测页面后,可以看到当前存储桶数量、对象数、用户数等数量信息,同时还可以查看服务器的状态、容量状况和网络连接等状态信息。管理员可以通过这些信息,随时监测系统的整体运行情况,并及时进行维护和管理。

\begin{figure}[htb]
    \centering
    \includegraphics[width=1\textwidth]{my_figures/chapter6/事件跟踪.png}
    \caption{事件跟踪}
    \label{fig:事件跟踪}
%     \note{注:图注的内容不宜放到图题中。}
\end{figure}

如图\ref{fig:事件跟踪}所示,进入系统监控的时间跟踪页面,可以查看近期发生的事件信息,包括操作名称、操作时间、操作状态及IP地址等。管理员可以通过这些事件信息及时掌握操作的准确情况,发现并跟踪异常操作(例如状态码为404 Not Found),及时进行问题的定位和解决。

% 根据测试结果,系统监控模块各项功能正常,能够满足系统管理员在生产环境中进行状态监测、事件跟踪和告警通知方面的需求。

\section{非功能性测试}

在非功能性测试中,主要从性能、安全性和易用性三个方面对系统进行测试,性能、安全性和易用性是评估本系统的关键指标,它们保证了系统能够在高负载、高并发的场景下稳定高效的
运行,且不会丢失和损坏任何数据,用户在使用时也会感觉简单易用,提升操作体验。为了确保测试结果的准确性和可靠性,将采用一系列具体的测试指标和数据,以便对平台进行全面
的评估和对比。



\subsection{性能需求测试}

性能测试是一种软件测试方法,用于评估系统在不同负载和压力条件下的响应时间、吞吐量、并发用户数量等指标。对于本平台而言,性能测试的重点包括高并发访问和快速响应。
为了对本平台进行全面的性能测试,设计了多用户同时上传和下载文件测试、不同大小文件上传和下载测试。如表\ref{tab:性能测试用例表}是系统的性能测试用例表。

\begin{longtable}{|m{0.16\linewidth}|m{0.3\linewidth}|m{0.3\linewidth}|m{0.11\linewidth}|}

    \caption{性能测试用例表}\label{tab:性能测试用例表} \\
     \endfirsthead
     \multicolumn{4}{r}{ \bf{续表} }  \\
     \hline
     测试点   & 测试步骤                          & 测试预期             & 测试结果 \\
     \hline
     \endhead
     \hline
     用例编号  & \multicolumn{3}{l|}{TC06} \\
     \hline
     测试类型  & \multicolumn{3}{l|}{非功能性测试}                                 \\
     \hline
     测试目的  & \multicolumn{3}{l|}{测试系统的性能}                          \\
    %  \hline
    %  对应需求编号 & \multicolumn{3}{l|}{UC10} \\ \hline
    %  前提条件  & \multicolumn{3}{l|}{系统管理员已成功登录系统}                        \\
     \hline
     期望结果  & \multicolumn{3}{l|}{系统能保证稳定的性能}                           \\
     \hline
     实际结果  & \multicolumn{3}{l|}{测试成功}                                 \\
     \hline
     测试点   & 测试步骤                          & 测试预期             & 测试结果 \\
     \hline
     模拟高负载测试 & 使用 JMeter 创建线程组模拟网页在高负载下的性能
                    & 平台各模块的平均响应时间和吞吐量均在推荐范围内   & 测试通过,测试数据见表\ref{模拟高负载测试数据表} \\
     \hline
     不同大小文件上传和下载测试 & \parbox[t]{4.5cm}{\hangindent=1em\hangafter=1 1.选择不同大小的文件作为测试文件} \vspace{-0.8mm}
                    \newline \parbox[t]{4.5cm}{\hangindent=1em\hangafter=1 2.使用系统的上传接口,将测试文件上传到系统中,并记录上传所需时间} \vspace{-0.8mm}
                    \newline \parbox[t]{4.5cm}{\hangindent=1em\hangafter=1 3.使用系统的下载接口,将上传的测试文件下载到本地,并记录下载所需时间} \vspace{-0.8mm}
                    \newline \parbox[t]{4.5cm}{\hangindent=1em\hangafter=1 4.重复以上步骤多次,确保测试结果准确可靠} \vspace{-0.8mm}
                    & 不同大小文件上传、下载的时间均值应小于建议值   & 测试通过,测试数据见表\ref{不同大小文件上传和下载测试数据表} \\
     \hline
\end{longtable}

(1)模拟高负载测试

模拟高负载测试主要是测试系统在多个用户同时进行系统操作时的高负载场景下的处理速度和响应时间。使用 Jmeter 线程组功能,创建线程组,设置 600 个线程,循环 1 次,
对管控平台的权限列表、用户列表、策略列表、桶列表、状态监测、事件跟踪和告警通知的接口进行测试,测试数据如表\ref{模拟高负载测试数据表}所示。


\begin{center}
    \renewcommand\arraystretch{1.5}{
    \setlength{\tabcolsep}{5mm}{
        \begin{longtable}{|p{1.5cm}<{\centering}|p{0.8cm}<{\centering}|p{1.2cm}<{\centering}|p{1.2cm}<{\centering}|p{1.2cm}<{\centering}|p{1.2cm}<{\centering}|p{1.2cm}<{\centering}|}
		\caption{模拟高负载测试测试数据表}\label{模拟高负载测试数据表}\\
		\hline
        \bf{页面} & \bf{样本}& \bf{平均值}  & \bf{最小值} & \bf{最大值}& \bf{吞吐量}\\
        \hline
        \endfirsthead
     
        \multicolumn{6}{r}{ \bf{续表} } \\
        \hline
        \bf{页面} & \bf{样本}& \bf{平均值}  & \bf{最小值} & \bf{最大值}& \bf{吞吐量}\\
        \hline
        \endhead
        \endfoot
        % \bottomrule
        \endlastfoot
        权限列表& 600  & 87 & 6  & 532 & 382.6/sec\\
        \hline
        用户列表& 600  & 368 & 5  & 224 & 462.3/sec\\
        \hline
        策略列表& 600  & 175 & 3  & 425 & 443.8/sec\\
        \hline
        桶列表& 600  & 286 & 5  & 578 & 442.7/sec\\
        \hline
        状态监测& 600  & 364 & 8  & 592 & 532.6/sec\\
        \hline
        事件跟踪& 600  & 93 & 7  &456 & 632.5/sec\\
        \hline
        告警通知& 600  & 132 & 6  & 677 & 438.4/sec\\
        \hline
        总体& 4200  & 172 & 4  & 738 & 1438.4/sec\\
        \hline
         
	\end{longtable}}}
    \vspace{-1cm}
\end{center}

根据测试数据显示,多个页面的平均响应时间均低于400毫秒,同时吞吐量稳定在每秒1400左右,这证明了系统在性能方面表现出色,已经满足了系统的性能需求。

(2)不同大小文件上传和下载测试

不同大小文件上传和下载测试主要是测试系统在不同大小的文件上传和下载时的处理速度和响应时间。这样同样使用JMeter来创建线程去访问文件上传和文件下载接口,每次测试时,选
用不同大小的文件作为测试样本点,并记录文件上传时间和文件下载时间,在多次重复测试情况下,计算出这些时间的平均值,并与建议值进行比较,如果小于建议值,
这说明性能测试通过,测试数据如表\ref{不同大小文件上传和下载测试数据表}所示。

\begin{center}
    \renewcommand\arraystretch{1.5}{
    \setlength{\tabcolsep}{5mm}{
        \begin{longtable}{|p{0.8cm}<{\centering}|p{0.8cm}<{\centering}|p{1.8cm}<{\centering}|p{1.2cm}<{\centering}|p{1.8cm}<{\centering}|p{1.2cm}<{\centering}|}
		\caption{不同大小文件上传和下载测试数据表}\label{不同大小文件上传和下载测试数据表}\\
		\hline
        \bf{文件大小} & \bf{重复次数} & \bf{上传时间平均值}  &\bf{建议值}& \bf{下载时间平均值} & \bf{建议值}\\
        \hline
        \endfirsthead
     
        \multicolumn{4}{r}{ \bf{续表} } \\
        \hline
        \bf{文件大小}& \bf{上传时间均值}  & \bf{下载时间均值} & \bf{上传时间建议值}& \bf{建议值}\\
        \hline
        \endhead
        \endfoot
        % \bottomrule
        \endlastfoot
        50MB  & 500 & 2.6s  & <=3.2s &4.3s & <=5.1s\\
        \hline
        100MB  & 500 & 5.2s  & <=6.3s &8.4s & <=10.2s\\
        \hline
        500MB  & 500 & 20.5s  & <=25.7s &18.7s & <=22.9s\\
        \hline
        1GB  & 500 & 52.5s  & <=65.7s &48.7s & <=62.9s\\
        \hline
	\end{longtable}}}
    \vspace{-1cm}
\end{center}

根据测试数据显示,系统能够快速地上传和下载各种大小的文件,且其上传和下载的平均时间均小于建议时间。特别是在处理大文件时,系统表现出更高效的性能,实际时间和建议时间
之间的差距更加明显,这表明系统能够满足对大文件上传和下载速度的要求。

\subsection{安全性需求测试}

安全性测试是软件开发生命周期中的一个重要环节,其作用是评估系统的安全性和风险,并检测系统中存在的安全漏洞和潜在威胁。通过安全性测试,可以发现和修复系统中存在的安全
缺陷,提高系统的安全性和稳定性,避免盗号登录、未经授权的访问、SQL注入和上传木马文件等安全问题的发生。如表\ref{tab:安全性测试用例表}为系统的安全性测试用例表。

\begin{longtable}{|m{0.16\linewidth}|m{0.3\linewidth}|m{0.3\linewidth}|m{0.11\linewidth}|}

    \caption{安全性测试用例表}\label{tab:安全性测试用例表} \\
     \endfirsthead
     \multicolumn{4}{r}{ \bf{续表} } \\
     \hline
     测试点   & 测试步骤                          & 测试预期             & 测试结果 \\
     \hline
     \endhead
     \hline
     用例编号  & \multicolumn{3}{l|}{TC06} \\
     \hline
     测试类型  & \multicolumn{3}{l|}{非功能性测试}                                 \\
     \hline
     测试目的  & \multicolumn{3}{l|}{测试系统的安全性}                          \\
    %  \hline
    %  对应需求编号 & \multicolumn{3}{l|}{UC10} \\ \hline
    %  前提条件  & \multicolumn{3}{l|}{系统管理员已成功登录系统}                        \\
     \hline
     期望结果  & \multicolumn{3}{l|}{系统能有效应对常见的安全风险}                           \\
     \hline
     实际结果  & \multicolumn{3}{l|}{测试成功}                                 \\
    %  \hline
    %  测试点   & 测试步骤                          & 测试预期             & 测试结果 \\
     \hline
     用户登录安全测试 &  \parbox[t]{4.5cm}{\hangindent=1em\hangafter=1 1.使用浏览器开发者工具获取 Cookie 和 session 内容} \vspace{-0.8mm}  
     \newline \parbox[t]{4.5cm}{\hangindent=1em\hangafter=1 2.修改 Cookie 和 session 内容,尝试伪造用户身份进行登录,记录登录结果} \vspace{-0.8mm}
     \newline \parbox[t]{4.5cm}{\hangindent=1em\hangafter=1 3.使用 Brute Force 攻击工具尝试进行用户登录,记录测试结果} \vspace{-0.8mm}
     
     & \parbox[t]{4.5cm}{\hangindent=1em\hangafter=1 1.当 Cookie 和 session 内容被篡改时,系统能够检测到异常情况并及时发出警报,同时禁止攻击者进行登录操作} \vspace{-0.8mm}
     \newline \parbox[t]{4.5cm}{\hangindent=1em\hangafter=1 2.系统能够在 Brute Force 攻击下保持稳定,即便攻击者使用多次登录尝试,也不能成功登录系统} \vspace{-0.8mm}
     \newline \parbox[t]{4.5cm}{\hangindent=1em\hangafter=1 3.在进行以上测试过程中,不会对系统正常的登录流程和数据造成影响} \vspace{-0.8mm}   & 测试用例1521, 通过率100\% \\
     \hline
     URL访问未授权页面测试 &  \parbox[t]{4.5cm}{\hangindent=1em\hangafter=1 1.在浏览器地址栏中输入未授权页面的 URL 地址}  \vspace{-0.8mm} 
     \newline \parbox[t]{4.5cm}{\hangindent=1em\hangafter=1 2.检查系统是否出现访问被拒绝的提示信息} \vspace{-0.8mm}
     \newline \parbox[t]{4.5cm}{\hangindent=1em\hangafter=1 3.使用 Burp Suite 等工具进行手动拦截请求并修改请求参数,尝试绕过权限控制访问未授权页面} \vspace{-0.8mm}
     \newline \parbox[t]{4.5cm}{\hangindent=1em\hangafter=1 4.检查系统是否成功拦截了未授权访问请求,即是否出现访问被拒绝的提示信息} \vspace{-0.8mm}
     & \parbox[t]{4.5cm}{\hangindent=1em\hangafter=1 1.对于未授权页面的 URL,系统的权限控制机制会限制访问} \vspace{-0.8mm}
     \newline \parbox[t]{4.5cm}{\hangindent=1em\hangafter=1 2.当攻击者尝试访问未授权页面时,系统应该拦截请求并显示访问被拒绝的提示信息,以防止攻击者获取未授权页面的敏感信息} \vspace{-0.8mm} & 测试用例1453,通过率100\%\\
     \hline
     文件上传测试 &  \parbox[t]{4.5cm}{\hangindent=1em\hangafter=1 1.尝试上传一个可执行脚本,例如以.js或.exe结尾的文件} \vspace{-0.8mm}  
     \newline \parbox[t]{4.5cm}{\hangindent=1em\hangafter=1 2.检查系统是否能够正确识别并拦截这个请求,防止文件上传} \vspace{-0.8mm}
     \newline \parbox[t]{4.5cm}{\hangindent=1em\hangafter=1 3.再次尝试上传一个普通的文件,例如.pdf或.jpg等} \vspace{-0.8mm}
     \newline \parbox[t]{4.5cm}{\hangindent=1em\hangafter=1 4.检查系统是否能够正确接收并保存这个文件} \vspace{-0.8mm}
     & \parbox[t]{4.5cm}{\hangindent=1em\hangafter=1 1.系统应该能够正确识别可执行脚本文件并拦截上传请求} \vspace{-0.8mm}
     \newline \parbox[t]{4.5cm}{\hangindent=1em\hangafter=1 2.应该无法成功上传可执行脚本文件} \vspace{-0.8mm}
     \newline \parbox[t]{4.5cm}{\hangindent=1em\hangafter=1 3.系统应该能够正确接收并保存普通文件} & 测试用例1579,通过率100\%\\
     \hline
     SQL注入测试 &  \parbox[t]{4.5cm}{\hangindent=1em\hangafter=1 1.在各列表的搜索框中输入以下字符:' OR 1=1;--'} \vspace{-0.8mm} 
     \newline \parbox[t]{4.5cm}{\hangindent=1em\hangafter=1 2.点击搜索按钮,查看系统的响应和反馈信息}
  
     & 系统应该会返回筛选的列表,而不是根据搜索关键字过滤出来的结果 & 测试用例1446,通过率100\%\\
   
     \hline
    
\end{longtable}


\subsection{易用性需求测试}

易用性测试主要是为了评估软件系统的用户界面和用户交互,以确定系统的易用性和用户友好性。这里主要针对一些常用的功能模块中的常用操作进行测试,让之前未接触本平台的50个用户
完成特定任务、寻找某些信息或完成某些操作,记录任务完成时间、错误率、用户满意度等。易用性测试数据如表\ref{易用测试数据表}所示。


\begin{center}
    \renewcommand\arraystretch{1.5}{
    \setlength{\tabcolsep}{5mm}{
        \begin{longtable}{|p{2.5cm}<{\centering}|p{2.4cm}<{\centering}|p{1.2cm}<{\centering}|p{2cm}<{\centering}|}
		\caption{易用性测试数据表}\label{易用测试数据表}\\
		\hline
        \bf{操作任务} &  \bf{任务完成时间}  &\bf{错误率}& \bf{用户满意度} \\
        \hline
        \endfirsthead
       
        \multicolumn{4}{r}{ \bf{续表} } \\
        \hline
        \bf{操作任务} &  \bf{任务完成时间}  &\bf{错误率}& \bf{用户满意度} \\
        \hline
        \endhead
        \endfoot
        % \bottomrule
        \endlastfoot
        创建策略并分配给用户  & 30s & 32.5\%  & 95.6\% \\
        \hline
        修改存储桶的生命周期  & 22s & 13.4\%  & 98.2\% \\
        \hline
        查看可用容量  & 5s & 2.6\%  & 100.0\% \\
        \hline
        查看事件分析报告  & 11s & 21.7\%  & 99.4\% \\
        \hline
        查看告警信息  & 6s & 1.2\%  & 99.8\% \\
        \hline
        上传文件后下载文件  & 16s & 32.5\%  & 96.5\% \\
        \hline
         
	\end{longtable}}}
    \vspace{-1cm}
\end{center}

根据测试结果显示,任务完成时间从5秒到32.5秒不等,其中创建策略并分配给用户和上传文件后下载文件的任务完成时间比较长;错误率从1.2\%到32.5\%不等,其中创建策略并分配
给用户和上传文件后下载文件的错误率比较高;用户满意度从95.6\%到100.0\%不等,其中查看可用容量的任务得到了最高的用户满意度。总体而言,系统中的多组用户能够在不超过30
秒的时间内完成各种任务,同时用户满意度平均值也达到了97\%以上。这表明系统的操作界面设计简单易用,用户能够快速上手,大大提高了用户的使用体验。

% 此外,系统界面的设计如图\ref{fig:系统界面设计图}所示。

\begin{figure}[htb]
    \centering
    \includegraphics[width=1\textwidth]{my_figures/chapter6/系统界面展示图.png}
    \caption{存储桶管理页面}
    \label{fig:系统界面设计图}
%     \note{注:图注的内容不宜放到图题中}
\end{figure}

如图\ref{fig:系统界面设计图}所示,存储桶管理页面的设计十分简洁,页面主要展示存储桶的名称、对象数和生命周期等关键信息,同时在最右侧展示了针对每个存储桶可以进行的操
作,用户能够快速方便地进行存储桶的管理。当用户需要创建新的存储桶时,设置访问策略不再需要填写复杂的策略配置信息,而是可以选择已经配置好的访问策略,大大缩短了创建流
程,节省了时间,存储桶创建界面如图\ref{fig:存储桶创建}所示。

\begin{figure}[htb]
    \centering
    \includegraphics[width=1\textwidth]{my_figures/chapter6/存储桶创建.png}
    \caption{存储桶创建}
    \label{fig:存储桶创建}
%     \note{注:图注的内容不宜放到图题中。}
\end{figure}

\section{本章小结}

本章节旨在对分布式对象存储系统管控平台进行功能性和非功能性测试,验证其可用性和效率。本次测试采用了黑盒测试和白盒测试相结合的方法,编写了详细的测试用例表,
并且展示了各项功能模块的测试结果。经过本次测试,可以验证该管控平台能够满足用户的需求,并提供较好的用户体验。总体而言,本次测试证明了本系统管控平台可靠且
高效,达到了预期的设计目标。
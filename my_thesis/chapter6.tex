\chapter{系统测试与分析}

% 本章首先对测试环境和测试工具进行了简单的介绍,然后根据第三章需求分析中所涉及到的功能性需求和非功能性需求分别进行测试,
% 对测试结果进行整理和分析,评估系统是否与预期的要求相符合。
本文先是对测试环境和测试工具做了简要的说明,接着又针对第三章需求研究中所讨论到的功能性需求与非功能性需求分别展开了试验,
并对测试成果加以了收集与分析,从而判断系统功能是否与所预期的需求相符合。

\section{测试环境与工具}

\subsection{测试环境}

该系统的生产环境是基于AliIO的分布式集群,由四台Linux主机构成,每台主机上都安装了AliIO,主机的配置环境如表\ref{主机配置环境表}所示。

\begin{center}
    \renewcommand\arraystretch{1.5}{
    \setlength{\tabcolsep}{5mm}{
	\begin{longtable}{|p{2.2cm}<{\centering}|p{3.4cm}<{\centering}|p{5.6cm}<{\centering}|}
		\caption{主机配置环境表}\label{主机配置环境表}\\
		\hline
        \bf{节点} & \bf{IP} & \bf{data} \\
        \hline
        aliio1 & 30.225.140.224 & /data/minio/data\\
        \hline
        aliio2 & 30.225.140.203 & /data/minio/data\\
        \hline
        aliio3 & 30.225.140.237 & /data/minio/data\\
        \hline
        aliio4 & 30.225.140.241 & /data/minio/data\\
        \hline
	\end{longtable}}}
\end{center}

分布式对象存储管理系统的测试环境为本地的个人电脑,其系统配置如表所示。

\begin{center}
    \renewcommand\arraystretch{1.5}{
    \setlength{\tabcolsep}{5mm}{
	\begin{longtable}{|p{2.2cm}<{\centering}|p{10cm}<{\centering}|}
		\caption{系统和硬件配置表}\label{系统和硬件配置表}\\
		\hline
        \bf{操作系统} & windows 10 \\
        \hline
        \bf{浏览器} & Chrome 91 \\
        \hline
        \bf{CPU} & Intel(R) Core(TM)  \\
        \hline
        \bf{内存} & 16GB 4267MHz DDR4 \\
        \hline
	\end{longtable}}}
\end{center}


\subsection{测试工具}

% 这里采用Postman和Jmeter对系统的稳定性进行测试, Postman是一种能够对接口进行测试的工具,它可以模拟客户端给服务器发送
% http请求,而且具有很强的兼容性,在很多操作系统均可使用,是开发人员和测试人员常用的调试和测试工具。JMeter主要是通过
% 模拟巨大负载对系统的强度和性能进行压力测试,此外,它还可以对应用程序进行功能测试和回归测试,通过脚本来判断程序是否
% 返回了预期结果。

我们可以使用Postman和Jmeter对整个系统的性能进行检测,Postman是一个能够直接对接口进行检测的方法,它能够模拟客户端直
接向服务器发送http请求,并且有很好的兼容性,在很多操作系统均可使用,是开发人员和测试人员常用的调试和测试工具。JMeter主
要是利用模拟大负荷系统的强度等特性的压力试验,此外,它还可以对应用程序进行功能测试和回归测试,通过脚本来判断程序是否返
回了预期结果。

我们可以使用Postman和Jmeter对整个系统的性能进行检测,Postman是一个能够直接对接口进行检测的方法,它能够模拟客户端直
接向服务器发送http请求,并且有很好的兼容性,在很多操作系统均可使用,是开发人员和测试人员常用的调试和测试工具。JMeter主
要是利用模拟大负荷系统的强度等特性的压力试验,此外,它还可以对应用程序进行功能测试和回归测试,通过脚本来判断程序是否返
回了预期结果。






\section{功能性测试}

分布式对象存储管理系统软件,是集文件存放功能与应用/系统管理为一身的综合系统。本系统目前主要有二种使用者,一类是存储应用的人员,另一类则是系统管理人员。


系统管理员工主要负责客户数据、系统分组、系统日志、系统状态、存取方式和集群扩容管理,因此他们还必须对已登录注册的客户和系统的分组进行控制。这类应用必须注册登记到系统管理再完成所有功能使用。通常,系统管理员工注册到系统管理时首先会检查用户列表、分组情况、存储使用情况和系统状态等。还应特别关注系统的告警信息,对系统进行及时的修复。对于用户访问策略以及分组
的访问策略也必须由系统管理员负责,主要是为了资源的隔离,用户的最终访问策略由用户个人访问策略和分组访问策略共同决定,即用户最终访问策略=用户访问策略+分组访问策略。

系统中的策略主要分为Bucket访问策略和用户访问策略。Bucket访问策略主要有public、custom和private三种,系统内置的用户访问策略主要有五种,分别是控制台管理员(consoleAdmin)策略、
诊断(diagnostics)策略、只读(readonly)策略、读写(readwrite)策略和只写(writeonly)策略。当设置Bucket的使用方式是public后,用户必须不通过任何验证才能进行使用资源;当设置Bucket的访问方式是custom后,Bucket的最终访问策略由具体的用户访问策略决定,用户访问策略可以是系统内定的策略,如只读(readonly)策略、读写(readwrite)策略和只写(writeonly)策略,
也可以是自定义的访问策略;当设置Bucket的访问策略为private时,用户未经授权不能进行任何操作,所有用户访问策略失效。因此,Bucket的访问策略权限要大于用户访问策略。

系统中暂时只有三个人物:特级经理,一般经理,还有一般用户。系统管理员也是目前唯一的超级管理者,掌握对操作系统中各种功能的运行权限,重点是能够检测服务器状态并且对系统进行修改或者扩展;一般管理者,掌握对使用者、分组和访问策略的管理权限,如查看用户信息和分组信息,对用户或分组进行禁用,修改用户或分组的访问策略;普通用户拥有对Bucket管理和文件上传下载
等权限

\section{非功能性测试}

分布式对象存储管理系统的非功能性需求,主要包括了效率高、一致性、扩展、简单以及安全,非功能性要求是保证整个系统完成质量
的重要基础,其在整个项目中具有与功能性要求同等的价值。要想将系统变成一种完整的、可以不断迭代的体系,就必须同时满足功能
性与非功能性的要求。





% 本管理系统的底层的AliIO服务器通过组合各种实例,形成了统一的全局名称空间,通常情况下,一个分布式模式集最多可以由32个AliIO服务器组合而成,
% 并且多个分布式模式集可以组合成一个AliIO服务器联合,每个AliIO服务器联合都提供统一的管理员和名称空间,而且它们所能支持的分布式模式集合是无限数量的。
% 采用这种方式后,那些地理上分散的大型企业可以通过这个对象存储进行大规模扩展,
本系统的最底层的AliIO系统通过整合各种实例,可以建立起共同的全局名字空间,在一般情况下,一个分布式模式集最多能够由32
个AliIO系统组合而成,同时多个分布式模式集也能够组成一个AliIO服务器组合,因此各个AliIO服务器组合都可以拥有共同的管
理员和名字空间,同时它所能够使用的分布式模板集也是无限多的。采取这个方法后,一些地域上分散的大型公司能够利用这个对象
存储实现规模扩张,同时可以容纳S3 Select,MinSQL,Spark等各种应用程序。


系统中的策略主要分为Bucket访问策略和用户访问策略。Bucket访问策略主要有public、custom和private三种,系统内置的用户访问策略主要有五种,分别是控制台管理员(consoleAdmin)策略、
诊断(diagnostics)策略、只读(readonly)策略、读写(readwrite)策略和只写(writeonly)策略。当设置Bucket的使用方式是public后,用户必须不通过任何验证才能进行使用资源;当设置Bucket的访问方式是custom后,Bucket的最终访问策略由具体的用户访问策略决定,用户访问策略可以是系统内定的策略,如只读(readonly)策略、读写(readwrite)策略和只写(writeonly)策略,
也可以是自定义的访问策略;当设置Bucket的访问策略为private时,用户未经授权不能进行任何操作,所有用户访问策略失效。因此,Bucket的访问策略权限要大于用户访问策略。



\section{本章小结}

本文先是对测试环境和测试工具做了简要的说明,接着又针对第三章需求研究中所讨论到的功能性需求与非功能性需求分别展开了试验,
并对测试成果加以了收集与分析,从而判断系统功能是否与所预期的需求相符合。
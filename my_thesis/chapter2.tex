\chapter{相关技术简介}

\section{分布式对象存储系统}

分布式对象存储系统(Distributed Object Storage System,简称DOSS)是一种新型的分布式存储系统,将数据以对象的形式存储在多个节点上,并通过对象标识符进行统一管理和访问。
它具有很好的可靠性、可用性、扩展性和数据处理能力,在互联网、金融、医疗、物联网等领域有广泛应用,是未来存储技术的重要发展方向。与传统的块存储和文件存储不同,DOSS具有以下特点:

\begin{figure}[h]
    \centering
    \includegraphics[width=0.8\textwidth]{my_figures/chapter2/存储模型对比图.png}
    \caption{对象存储模型与传统存储模型的区别}
    \label{fig:/存储模型对比图}
%     \note{注:图注的内容不宜放到图题中。}
\end{figure}

(1)对象存储:DOSS将数据以对象的形式进行存储,一个对象通常包括数据、元数据和对象标识符等信息,方便管理和访问。

(2)分布式存储:DOSS采用分布式存储方式,将数据分散存储在多个节点上,提高了数据的可靠性和可用性。

(3)异构性:DOSS支持多种数据类型和格式,如文本、图片、视频等,同时也支持多种协议和接口。

(4)高可靠性:DOSS采用副本机制和容错技术,保证数据的可靠性和安全性,即使某个节点出现故障,数据仍然可以被访问和恢复。

(5)高扩展性:DOSS采用分布式架构,可以根据实际需要添加或删除节点,从而实现扩展和收缩。

DOSS适用于需要存储海量数据的场景,同时具有很好的数据处理能力,支持数据分析、数据挖掘等操作,可以帮助企业更好地发掘数据价值。对象存储模型与传统存储模型的区别如图\ref{fig:/存储模型对比图}所示。




\section{分布式对象存储系统管控平台}

分布式对象存储系统(DOSS)管控平台是一种用于管理和控制DOSS集群的管理软件,它可以实现DOSS的统一管理、监控、调度和优化,提高DOSS的可用性、可靠性和性能。

DOSS管控平台通常包括以下几个方面的功能:

(1)集群管理:用于管理DOSS集群的拓扑结构、节点状态、数据分布等信息,支持节点的添加、删除、扩容、缩容等操作,实现集群的动态管理和调整。

(2)数据管理:用于管理DOSS存储的数据对象,包括对象的上传、下载、删除、复制、迁移等操作,支持数据的版本控制、回收站、检索等功能,提高数据的管理效率和可用性。

(3)性能监控:用于监控DOSS集群的性能和健康状况,包括节点的负载、网络延迟、数据吞吐量等指标,支持实时监控和历史数据的统计和分析,帮助管理员快速发现和解决故障。

(4)安全管理:用于管理DOSS集群的安全性,包括用户认证、访问控制、数据加密等功能,支持角色管理、审计日志等功能,保护DOSS存储的数据不被恶意访问或篡改。

(5)优化调整:用于优化DOSS集群的性能和资源利用率,包括节点负载均衡、数据分片调整、网络流量控制等功能,支持自动化调整和手动调整,提高DOSS集群的运行效率和稳定性。

总之,DOSS管控平台是一种重要的软件工具,可以帮助DOSS管理员管理和控制DOSS集群,提高DOSS的可用性、可靠性和性能,是DOSS技术的重要组成部分。

\section{AliIO}

AliIO是一种高性能、分布式的对象存储系统,与传统的存储以及其他类型的对象储存技术相较而言,其最大不同之处就是:它从一开始就按照针对性能需求更高的私有云规范完成了软件架构工程设计,其简易架构如图\ref{fig:/AliIO简易架构图}所示。由于AliIO选择了更简易的方法来实现产品设计,它就可以完成传统对象储存技术所需要的所有功能,从而具备了高效的特性,而且也不会因为更多的业务功能而影响AliIO的易用性和高效性。这样的设计结果所带来的最大优势就是:它可以更容易的完成原生对象储存业务,而且还具有弹性的伸缩能力。因此AliIO在辅助存储,灾难修复以及归档等传统对象储存应用案例方面都体现的非常优秀。同时,它在机器学习、大数据分析、私有云、混合云等方面的对象储存技术上也独具一格。当然,在大数据分析、高性能应用负载、原生云的支持等方面也发挥着巨大的优势。
其主要特性如下:

\begin{figure}[h]
    \centering
    \includegraphics[width=0.8\textwidth]{my_figures/chapter2/AliIO简易架构图.png}
    \caption{AliIO简易架构图}
    \label{fig:/AliIO简易架构图}
%     \note{注:图注的内容不宜放到图题中。}
\end{figure}


(1)高性能:拥有优异的读取特性,读出的速率高达185 GB/秒,而输入速率则高达175 GB/秒。对象存储可用作主内存层,主要用来管理Spark、Presto、TensorFlow、H2O.ai及其各类重复工作负载,而AliIO在云原生方面,具有比传统对象更高效的特性,能够适应云原生应用程序对高速吞吐和低延时的要求。

(2)可扩展性:采用了Web缩放器的设计,为数据存储创造了一个更轻松的缩放模式。AliIO从单个集群开始的拓展,这个集群能够通过和其他AliIO集群结合来产生全局名字空间,并且在使用中能够跨越几个不同的数据中心。而随着集群的加入,更多集群计算机也能够对名字空间进行拓展。

(3)云原生支持:和所有原生云计算技术的基本框架和构建过程相符合,把云计算技术的新技术和概念都加入到了其中,如支持Kubernetes、微服务和多租户的容器技术。

(4)简单易用:AliIO的指导性产品设计理念为极简主义。简化的设定方法降低了错误的机率,也增加了正常工作时间,从而提高了安全性,同时简易度也是系统稳定性的基石。配置与使用AliIO的流程都非常简洁,只需加载一个二进制软件然后运行,在数分钟内就能够实现。同时设置选项的数量也非常少,能够使设定错误的机率减至极少的程度,几乎不会出错。升级过程中也不需要使用简单的命令就可以实现,并且升级后不需要重启就能生效,极大降低了使用成本和运维成本。

(5)与Amazon S3兼容:亚马逊云的S3 API(接口协议)是在世界范围内达成共识的对象存储的协定,是全球认可的准则。




\section{微服务}


微服务架构是一种面向服务的架构风格,将应用程序拆分为一组松散耦合的、可独立部署和扩展的服务。每个服务都有自己的独立进程,可以使用不同的编程语言、数据库、技术堆栈等。微服务架构的优点包括:

(1)独立部署和扩展:每个服务都是独立的,可以单独部署和扩展,不会影响其他服务。

(2)松散耦合:服务之间的耦合度低,可以更容易地进行修改、重构和升级,不会影响整个系统。

(3)可组合性:每个服务都可以作为一个独立的组件,可以组合在一起构建更复杂的应用程序。

(4)容错性:如果一个服务失败,其他服务仍然可以正常运行,系统整体的容错性更强。

(5)独立团队开发:每个服务可以由不同的团队独立开发和维护,可以更快地推出新功能和修复问题。

本系统采取了微服务结构,把核心服务模块与非核心服务模块进行了分开,将系统各部分完全解耦,从而使得系统中整个的服务结构更加清晰,也减少了过多冗余的工作,维护起来也更加简单,从而使整个系统可以健康高速的发展迭代。

\section{后端开发框架}

\subsection{SpringBoot}


Spring Boot是在 Spring框架的基础上发展而来,相比 Spring而言,SpringBoot省去了繁琐的配置过程,开发时快速高效。
Spring框架作为一个出色的开源应用架构,给后端开发人员带来了良好的解决方案,而控制反转也是它最大的特点所在,该特性
可以通过内置容器的设计思想对对象生命周期进行有效的管控。开发者可以使用Spring所规定的的格式撰写配置文件和配置类的
相关消息,将管理对象的任务由开发人员直接传递至Spring框架的内置容器。在程序启动后,开发者可以使用以下两类方法获得由
Spring托管的管理对象,一类方法是通过依赖注入,另一种是调用内置容器接口。本项目是使用Spring Boot的一个微服务。


\subsection{Spring Cloud Gateway}

Spring Cloud中集成了多个框架,通过这些架构可以共同处理微服务架构中存在的各种问题,因此Spring Cloud为开发者提供了一个优秀而完备的解决方案,开发者也可以按照自身的实际需要应用这些模块。

\begin{figure}[h]
    \centering
    \includegraphics[width=0.8\textwidth]{my_figures/chapter2/网关在微服务中的位置图.png}
    \caption{网关在微服务中的位置图}
    \label{fig:/网关在微服务中的位置图}
%     \note{注:图注的内容不宜放到图题中。}
\end{figure}

Spring Cloud Gateway是Spring Cloud中最常见的一种功能,它的主要功能是协助开发者迅速而简单的建立一个多功能的API网关服务, API网关功能就相当于微服务的门户, API的转发和路由都可以在这个功能中实现,并且针对指定类型的API还可以按照需求进行一些鉴权或者限流的动作[23]。

而本项目的微服务 API 网关就是通过Spring Cloud Gateway框架进行构建的,先对用户的请求进行鉴权,然后再以请求的URI为基础把请求路由到特定的微服上。网关在微服务中的位置如图\ref{fig:/网关在微服务中的位置图}所示。



\section{本章小结}

本章首先介绍了分布式对象存储技术的组成和特性。之后阐述了本系统所采用的分布式对象存储系统AliIO的总体概况与关键技术特点,AliIO是由阿里云自行研发的一种高性能的分布式对象存储系统软件,是本系统设计的重要依据。接着介绍了分布式存储系统管控平台,它是一种用于管理和控制 DOSS 集群
的管理软件。最后引入了微服务框架、后端框架Spring Boot和SpringCloud Gateway,Spring Boot为程序员发布单个微服务带来了很大的方便,而SpringCloud Gateway则作为API网关很好的保护了微服务,同时也对与核心服务无关的系统维护功能进行隔离,从而使系统更加可靠且更易于维护。


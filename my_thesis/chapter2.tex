% !TeX root = ../main.tex

\chapter{相关技术简介}
% 本章将介绍与分布式对象存储管理系统相关的技术,首先介绍了对象存储系统的组成和特性,然后介绍对象存储系统AliIO的特性,以及
% 底层存储设备AliFlash特点,最后对微服务架构和使用的后端框架进行了简述。
本文将讲述与分布式对象存储器系统有关的新技术,首先讲述了对象存储系统的基本构造与特点,接着讲述了对象存储系统AliIO的基本特点

\section{对象存储系统}


% 对象存储系统是一种用户、元数据服务器和存储设备 OSD通过网络相互链接起来的分布式文件系统。其中,元数据服务器的主要作用是对全局的命名空间进行维护,同时管理所存储的
% 文件和映射关系,对于外界的访问设置身份验证等访问权限来保证系统安全。用户需要对系统进行管理时,只需要调用相应的接口即可实现,在系统内部,基本存储单元是对象。
 

% 通常一个对象存储系统由以下几个部分组成:


% 1.对象。对象分别由对象标识、数据和对象的属性三部分组成,它是对象存储系统的独立存储单位。其中对象的属性主要包括对象的基本特征、读写操作权限以及存储的数据占比情况
% 等信息。每一个存储在系统中对象都会有唯一的标识。

% 2.基于对象的存储设备 OSD。其功能和当今的磁盘驱动器十分相似,但和磁盘驱动器相比,OSD更加智能。OSD可以通过存储的所有对象属性信息对存储在本地的对象进行简单的管理。
% 由于将属性信息从元数据服务器中剥离出来,存储到了对象存储设备中,这样大幅度减轻了元数据服务器的负担,单点失效问题在一定程度上也会得到缓解。

% 3.元数据服务器。元数据服务器是对象存储系统的重要组成部分,主要用来存储对象的元数据,为外界提供元数据相关的功能。此外,元数据服务器可协调和控制分布在整个集群中的节点,有利于数据的共享,集群节点
% 的分布式一致性也会得到提升。


% 4.对象存储文件系统。系统对对象的生命周期进行管理时,主要包括创建和拷贝,以及保持整个集群的负载均衡等。系统具有一定的容错能力,对于存在的问题可以进行自我修复,
% 并对系统的操作做出响应。将系统内的各个对象都聚集在一起,对每个对象的访问权限进行有效的控制。系统对外提供了很多接口,比如文件的读取、写入和删除等都有对应的
% 操作接口。


% 对象存储系统中,会将数据对象和数据属性分开进行操作,对于数据对象的操作主要是写入对象或读取对象等。采用基于对象的存储设备来搭建对象存储系统是很关键的,因为对象
% 存储设备具备智能化的管理的能力,可以对存储在系统中的每个对象在整个集群架构中的分布进行智能化的管理。



% 对象存储(Object-Based Storage)是目前网络存储领域的热门研究方向,它是一种基于对象的新兴存储技术,是构建大型分布式系统的优秀选择方案。在功能特性上,它同时具备
% NAS(Network Attached Storage)[25]和 SAN(Storage Area Network)[26]的优势,实现了高性能、跨平台与数据安全共享的存储结构。互联网上的存储设备和存储组件
% 通过对象存储技术连接在一起,然后利用一定的技术实现存储空间的共享,让伸缩性和吞吐量都得到了极大的提升。在数据文件的归档以及非结构化数据的存储和管理方面都有极大
% 的优势,存储管理的成本显著降低。另外,对象存储技术的应用十分广泛,如卫星领域、医学视频领域的数据处理、数据仓库等领域都有实际应用。


% 对象存储的特性可归纳为以下几点: 

% 1.数据共享。对象是对象存储的基本组成单位,每个对象都有各自的属性,属性的内容包括访问操作类型、生成时间等,对象的接口和基于文件的接口十分相似,不同点在于基于对象
% 的接口可以支持跨平台共享数据,因此,对象存储技术可以支持数据共享。



% 2.安全性。对象存储比块存储具有更好的安全性能。传统的块存储对网络环境和应用环境都有安全方面的要求,而对象存储则不需要严格要求安全的应用和网络环境。对象存储技术
% 的基本单位是对象,安全机制可以进一步细化到对象上,这样更加灵活有用。

% 3. 智能化。对象存储技术可以对内部存储的对象进行自主管理。在对象存储系统中,对象存储设备可以实现数据的智能存储和安全访问。

数据存储系统,是指一个将数据、元数据服务器和存储设备OSD,透过计算机互相联系开来的分布式文件管理系统。这里,元数据服务器的主要功能就是对全局的命名空间进行维护,利用同时管理所存的元数据和映射关系,针对外界的存取和进行身份验证的存取能力,来维护系统安全。如果用户要对系统进行控制的话,只需使用相应的接口技术就能够完成,在控制系统中,以基本的存储单元为对象。

通常一个对象存储系统由以下几个部分组成:
1.对象。对象分别由对象标志、数据类型和对象的属性三个部分所构成,它们都是对象存储系统的独立存储单元。其中对象的属性信息主要包含了对象的基本特征、读写操作权限和内存的数据占比状况等信息。每一种存储方式在整个系统中,对象都会有唯一的标识。

2.基于对象的存储设备OSD。其作用原理与当今的传统磁盘驱动器十分相似,但原理与传统磁盘驱动器一样,OSD更加智能。OSD还能够利用系统内存的所有对象属性信息,对存放在本地空间的所有对象进行简单的管理。由于将属性信息从元数据服务器中剥离了开来,并保存在了对象存储器中,这样大幅度降低了元数据服务器的压力,因此单点失效问题在一定程度上也就会有所减轻。

3.元数据服务器。元数据服务器是对象存储系统的关键部分,主要用于保存各种数据的元数据内容,为外界实现元数据服务的能力。另外,元数据服务器还能够协调和管理散落在整体数据集群上的节点,有利于用户的资源共享,而数据集群节点的分布式一致性也将会有所增强。

4.对象存储文件系统。当系统对对象的整个生命周期进行管理时,重点涉及对象创建和拷贝,以及维护对整个集群的负载均衡等。控制系统也具备了一定的控制技术能力,对出现的问题能够实现自主修复,并对整个系统的正常运行进行了响应。把整个系统内的所有对象都集中到一起,对各个对象的存取权限实现了合理的管理。同时系统还对外提供了许多接口,包括对文档的阅读、输入和删除等均具有相应的操作界面。

数据存储系统中,主要是将数据中的数据属性分别加以使用,对于数据属性的作用主要有输入对象或读出对象等。通过一个对象的存储设备来搭建数据存储
系统管理也是非常重要的,由于对象存储设备系统具有智能的集中管理的能力,可通过对存放在系统中的各种对象在整个集群结构中的分配,实现智能的集中管理。
对象储存信息技术(Object-Based Storage)是目前国际网络储存领域的热门研究方向,它是一项依托对象的新型储存信息技术,是构建大型分布式系统的优秀选择方案。在功能特性上,它同时具备NAS(Network Attached Storage)[25]和SAN(StorageArea Network)[26]的优点,从而达到了高性能、跨平台和数据资源共享的储存架构。将网络系统中的数据与存储模块通过使用对象储存信息技术相互链接到一起,再通过一定的方式进行存储空间的共享,从而让伸缩性与吞吐量都获得了很大的提高。在大数据文件的归档和非结构化信息的保存与处理上也有很大的优点,存储管理的成本显著降低。此外,对象存储技术的运用也非常普遍,如卫星应用领域、医疗视频应用领域中的信息处理、数据仓库等领域都有实际应用。

数据存储的功能可以概括为以下几点:

1.数据共享。对象是对象信息储存的基础构成单元,各个对象都具有相应的自身属性,自身属性的具体内容一般包含了访问操作类别、生成时间等,对象的连接也与基于文本的连接十分相似,不同之处就是通过面向对象的连接方式能够支撑跨平台共享数据,所以,通过面向对象存储技术也能够支撑资源共享。

2.安全性。对象存储器和块数据库都有良好的安全特性。一般的块储存技术对于网络环境和应用环境都有安全性层面的需求,而对象储存技术则不存在严格要求安全性的应用和网络环境。由于面向对象储存技术中的基础单位都是对象,因此安全机制也可逐步细分到对象上,这样就比较方便实用。

3.智能化。数据存储技术,能够对对象内部所存放的数据实现自主控制。在对象存储机制中,通过数据存储能够进行对数据的智能保存与安全存取。

\section{AliIO}



% AliIO 是一款高性能、分布式的对象存储系统. 它是一款软件产品, 在诸如X86等低成本机器这类标准硬件上也能够很好的
% 运行AliIO。AliIO与传统的存储和其他的对象存储相比而言,其不同点在于:它一开始就针对性能要求更高的私有云标准进行软件架
% 构设计。AliIO采用了更易用的方式进行设计,它能实现对象存储所需要的全部功能,具有很高的性能,它没有为了更多的业务功
% 能而妥协而影响AliIO的易用性和高效性。 这样的结果所带来的好处是:它能够更简单的实现原生对象存储服务,且该服务局有弹性伸
% 缩能力。
% AliIO在辅助存储,灾难恢复和归档等传统对象存储用例方面都表现的极为出色。同时,它在机器学习、大数据、私有云、混合云等
% 方面的存储技术上也独树一帜。当然,在数据分析、高性能应用负载、原生云的支持等方面也发挥着巨大的优势。

% AliIO的特性如下:

% 1.高性能。AliIO 具有良好的读写性能,读取的速度可达185 GB / 秒,写入速度可达175 GB / 秒。对象存储可作为主存储层,主要用于处理Spark、Presto、TensorFlow、
% H2O.ai等各种复杂工作负载,AliIO在云原生方面,拥有比传统对象更高的性能,可以满足云原生应用程序高吞吐两和低延迟的需求。


% 2.可扩展性。AliIO利用了Web缩放器的原理,为对象存储提供了简单的缩放模型。AliIO从单个集群开始进行扩展,该集群可以与其他AliIO集群联合以创建全局名称空间, 
% 并在需要时可以跨越多个不同的数据中心。 通过添加更多集群可以对名称空间进行扩展。

% 3.云的原生支持。AliIO与一切原生云计算的架构和构建过程符合,将云计算的全新技术和概念融入到其中,如支持Kubernetes 、微服务和多租户的容器技术。


% 4.简单。AliIO的指导性设计原则是极简主义。简单的设计减少了出错的机会,提高了正常运行时间,增加可靠性,同时简单性又是性能的基础。 安装和配置AliIO的过程十分
% 简单,只需要下载一个二进制文件然后执行,在几分钟内即可完成。配置选项的数量也很少,可以将配置失败的概率降到非常低的水平,几乎不会出错。升级过程也只需要通过
% 简单的命令即可完成,并且升级后不需要重启就能生效,极大降低了使用成本和运维成本。

% 5.与Amazon S3兼容。亚马逊云的 S3 API(接口协议) 是在全球范围内达到共识的对象存储的协议,是全世界公认的标准。

AliIO是一种高性能、分布式的数据存储系统.这也是一种软件产品,在像<br>X86的低成本计算机这类的系统上,都可以很好的运用AliIO。

AliIO和传统的存储以及其他类型的对象储存技术相较而言,其最大不同之处就是:它从一开始时就按照针对性能需求更高的私有云规范完成了软件架构工程设计。由于AliIO选择了更简易的方法来实现产品设计,它就可以完成传统对象储存技术所需要的所有功能,从而具备了很高效的特性,而且也不会因为更多的业务功能而妥协,而影响AliIO的简洁易用以及高效度。而这样的设计结果所带来的最大优势就是:它可以更容易的完成原生对象储存业务,而且该业务管理局还具有弹性的伸缩能力。因此AliIO在辅助存储,灾难修复以及归档等传统对象储存应用案例方面都体现的非常优秀。同时,它在机器学习、大数据分析、私有云、混合云等方面的对象储藏技术上也独具一格。当然,在大数据分析、高性能应用负载、原生云的支持等方面也发挥着巨大的优势。

AliIO的特性如下:

1.高性能。AliIO拥有优异的读取特性,读出的速率高达一百八十五GB/秒,而输入速率则高达一百七十五GB/秒。对象存储可用作主内存层,主要用来管理Spark、PrestO、TensOrFlow、H2o.ai及其各类重复工作负载,而AliIO在云原生方面,具有比传统对象更高效的特性,能够适应云原生应用程序对高速吞吐二和低延时的要求。

2.可扩展性。AliIO采用了Web缩放器的设计,为数据存储创造了一个更轻松的缩放模式。AliIO从单个集群开始的拓展,这个集群能够通过和其他AliIO集群结合来产生全局名字空间,并且在使用中能够跨越几个不同的数据中心。而随着加入,更多集群计算机也能够对名字空间进行了拓展。

3.云的原生支持。AliIO和所有原生云计算技术的基本框架和构建过程相符合,把云计算技术的新技术和概念都加入到了其中,如支持Kubernetes、微服务和多租户的容器技术。

4.简单。AliIO的指导性产品设计理念为极简主义。简化的设定方法降低了错误的机率,也增加了正常工作时间,从而提高了安全性,同时简易度也是系统稳定性的基石。配置与使用AliIO的流程都非常简洁,只需加载一个二进制软件然后运行,在数分钟内就能够实现。同时设置选项的数量也非常少,能够使设定错误的机率减至极少的程度,几乎不会出错。升级过程中也不需要使用简单的命令就可以实现,并且升级后不需要重启就能生效,极大降低了使用成本和运维成本。

5.与Amazon S3兼容。亚马逊云的S3API(接口协议)是在世界范围内达成共识的对象存储的协定,是全球认可的准则。



\section{AliFlash ZNS}




% AliFlash ZNS是阿里云自研存储团队在发布了业界第一版企业级Open Channel SSD产品规范之后自主开发的存储新硬件。
% ZNS(Zoned Namespace,分区命名空间)是从 OC(Open Channel,开放通道)SSD基础上衍生而来的,它将FTL
% (Flash Translation Layer,闪存转换层)从SSD内部迁移到上层的Host端,让Host端直接接触到SSD的内部,这样一来用户就可以根据自己需要,比较灵活的拥有自己特定的FTL
% ,这么做的代价就是必须重新设计软件架构,这需要花费高额的成本,也对用户的技术实力提出了很高的要求。ZNS是Open Channel技术的迭代演进,相比OC,通过Zone对于底层介质物理地址的抽象,在牺牲了部分对于介质底层掌握度的同时,降低了主机端软件的复杂度。
AliFlash ZNS是阿里云自研存储团队在发布了业界第一版企业级OpenChan-nel SSD产品规范之后自主开发的存储新硬件。ZNS(Zoned Namespace,分区命名空间)是从OC(Open Channel,开放路径)SSD基石上衍生而来的,它将FTL(Flash Translation Layer,闪存转换层)从SSD底层转移到了上面的Host端,让Host 
端可以接触到SSD的内部,这样客户就能够按照自身需求,更加方便的使用自己专属的FTL,但这样做的代价就是需要重新设置软件架构,这需要花费高额的成本,也对用户的技术实力提出了很高的要求。ZNS是 Open Channel技术的迭代演进,相比 OC,通过 Zone对于底层介质物理地址的抽象,在牺牲了部分对于介质底层掌握度的同时,降低了主机端软件的复杂度。


AliFlash ZNS的特性如下:

1.Write Order保序。把一张完整的SSD盘(namespace)按照LBA地址分段,划分成若干个Zone,Zone内顺序写,Zone间可乱序写。同时,AliFlash ZNS盘支持同一个Open Zone在同一个IO Queue内同时下发多条写命令保序,主机端下发的写命令不能跨Zone。

2.支持两条写命令的LBA Overwrite。主机端在发送写命令的时候可以进行前后命令的4K覆盖写,同时在同一个Zone内不会有数量或者边界的限制。而且Open Zone的最后一个4K也要覆盖写,不会在Zone写满之后固件就直接刷新到nand上去。

3.NAND介质可靠性强化保障。盘片会保存并周期性的检查每个Zone的关闭时间,如果关闭时间超过阈值则选为GC Victim。盘片会监控主机下发的读命令,并保存和更新每个Zone的读计数器,且做周期性的检查。如果读计数器超过阈值则选为GC Victim。盘片会维护元数据,元
数据中保存着每个Zone的关闭时间、读计数器、Zone映射表和异常掉电情况下的元数据保存和恢复。



\section{微服务}


% 微服务是一种软件架构设计风格, 与之相对应的是单体服务架构。 一个大型的、 功能复杂的系统应该按照功能模块将它拆分为多个微服务。 整个应用中的单个微服务可被单独部署, 独立运行。 各个微服务之间耦合度较低, 可以使用各种形式的通信协议进行交互。 每个微服务专注于完成
% 一个特定的任务, 这样整个系统没有过多冗余的功能或代码 [15]。

微业务是一个软件架构的风格,与它相对应的是单体软件结构。一个庞大的、内容繁杂的体系可以通过功能模块把其拆分成若干个微型
业务。一个应用中的一个微服务都可以被独立配置,或者单独工作。因为所有的微服务间耦合度都很小,所以能够通过不同类型的通信协
议实现信息交换。因为所有微服务都专注于做某个专门的工作,所以整个系统中不会有太多冗余的模块或程序[15]。

微服务架构有以下特性:

% 1.细粒度的服务分解: 每个微服务都具有很小的服务粒度,单个微服务都只针对单一的业务,方便程序员进行维护和复用。

% 2.技术栈可自由选择: 可以采用不同的语言和技术栈实现单个微服务,不同的微服务之间的消息传递依靠restful风格的API完成,此类API可以采用不同的技术实现。所以每个服务
% 采用的技术选型可以不尽相同,数据库软件和中间件也可灵活搭配。

% 3.独立部署运行: 单个微服务可以在一个进程或容器中独立部署,微服务之间彼此隔离,只有在运行时才会产生联系,这种部署方式可以实现快速交付。

% 本系统采用微服务架构, 将核心业务模块和非核心业务模块进行分离,系统各部分充分解耦,使系统整体的架构变得清晰,也避免了过多冗余的工作,维护起来也更方便,使得整个
% 系统能够健康快速的迭代[18]。

1.小粒度的服务分解:所有微服务都拥有最小的业务粒度,而每个微服务都只针对独立的服务,以方便于程序员的操作与复用。

2.技术栈可以选择:可以通过不同的语言或者技术栈完成同一个微业务,而不同的微业务间的消息传递可以依靠restful风格的API实现,而这些API也可能通过不同的技术实现。所以每个服务应用的技术选择都可能多种多样,而数据库软件与中间件也可以灵活配合。

3.独立部署业务:单个微业务能够在同一个过程或容器上独立部署工作,微业务间相互隔绝,只在执行时才会发生连接,这种部署模式能够做到更高效交付。

本系统采取了微服务结构,把核心服务模块与非核心服务模块进行了分开,将系统各部分完全解耦,从而使得系统中整个的服务结构更加清晰,也减少了过多冗余的工作,维护起来也更加简单,从而使整个系统可以健康高速的发展迭代[18]。

\section{后端开发框架}

\subsection{SpringBoot}

% Spring Boot 是在 Spring 框架的基础上发展而来,相比Spring而言,SpringBoot省去了繁琐的配置过程,开发时快速高效。Spring框架作为一款优秀的开源应用框架,为后端
% 开发者提供了优秀的解决方案,控制反转就是其最大的特性之一,该特性使用内置容器的思想对对象生命周期的进行管理。开发人员通过Spring所要求的的格式填写配置文件和配置
% 类的有关信息,将管理对象的任务由开发人员转移至Spring 框架的内置容器。程序启动后,开发人员可通过两种方式获取Spring托管的对象,一种是依赖注入,另一种是调用内置
% 容器接口。此项目使用 Spring Boot 开发单个微服务。

Spring Boot是在 Spring框架的基础上发展而来,相比 Spring而言,SpringBoot省去了繁琐的配置过程,开发时快速高效。
Spring框架作为一个出色的开源应用架构,给后端开发人员带来了良好的解决方案,而控制反转也是它最大的特点所在,该特性
可以通过内置容器的设计思想对对象生命周期有效的进行管控。开发者可以使用Spring所规定的的格式撰写配置文件和配置类的
相关消息,将管理对象的任务由开发人员直接传递至Spring框架的内置容器。在程序启动后,开发者可以使用以下二类方法获得由
Spring托管的管理对象,一类方法是通过依赖注入,另一种是调用内置容器接口。本项目使用Spring Boot的一个微服务。


\subsection{Spring Cloud Gateway}

% Spring Cloud 集合了众多框架, 这些框架共同解决微服务架构中出现的一系列问题, Spring Cloud 为开发人员提出了一套优秀且完善的解决方案, 开发人员可根据自己
% 的需求使用这些组件。

% Spring Cloud Gateway 是 Spring Cloud 中常用的一个组件, 它的主要作用是帮助开发人员快速便捷的构建一套多功能的 API 网关服务, API 网关服务相当于微服务的门户,
% 是所有微服务的入口, API的转发和路由都在这个服务中完成, 对于特定的 API会根据需要实现一些鉴权和限流等操作[23]。

% 在此项目的微服务 API 网关就是采用 Spring Cloud Gateway 框架来搭建的, 对用户的请求进行鉴权, 并以请求的URI为依据将请求路由到指定的微服务中。

Spring Cloud中集成了多个框架,通过这些架构可以共同处理微服务架构中存在的各种问题,因此Spring Cloud为开发者提供了一个优秀而完备的解决方案,开发者也可以按照自身的实际需要应用这些模块。

Spring Cloud Gateway是Spring Cloud中最常见的一种功能,它的主要功能就是可以协助开发者迅速而简单的建立一个多功能的API网关服务, API网关功能就相当于微服务的门户,是整个微服务的门户, API的转发和路由都可以在这个功能中实现,并且针对指定类型的API还可以按照需求进行一些鉴权或者限流的动作[23]。

而本项目的微服API系统则是通过Spring Cloud Gateway框架进行构建的,先对用户的请求进行鉴权,然后再以请求的URI为基础把请求路由到特定的微服上。

\section{本章小结}

% 本章首先介绍了对象存储技术的组成和特性。之后介绍了本系统所使用的分布式对象存储系统AliIO的整体概述和技术特性,AliIO是阿里自研的一款高性能的分布式对象存储系统软件,是本系统构建的基础。接着介绍了本系统所使用的存储硬件AliFlash ZNS,AliFlash ZNS完全由阿里自研,在硬件上进一步提升了存取性能。最后介绍了微服务框架、后端框架
% SpringBoot和SpringCloud Gateway,Spring Boot 给程序员开发单个微服务提供了极大的便利,而SpringCloud Gateway作为 API 网关很好的保护了微服务, 并且把与核心业务无关的系统维护功能分离开来, 增强了整个系统的可靠性和可维护性。

本章首先介绍了对象存储技术的组成和特性。之后阐述了本系统所采用的分布式数据保存体系及AliIO的总体概况与关键技术特点,AliIO是由安莱公司自行研发的一种高性能的分布式数据存储体系软件,是本系统建设的重要依据。接着介绍了本系统所使用的存储硬件AliFlash ZNS,AliFlash ZNS完全由阿里自研,在硬件上进一步提升了
存取性能。最后引入了微服务框架、后端结构的Spring Boot和SpringCloud Gateway,Spring Boot为程序员发布单个微业务带来了很大的方便,而SpringCloud Gateway则作为API网关很好的维护了微业务,同时也将和核心服务无关的系统维护功能隔离起来,从而提高了完整系统的稳定性和可保护。


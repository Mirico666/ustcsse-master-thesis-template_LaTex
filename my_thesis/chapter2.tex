\chapter{相关技术简介}

\section{分布式对象存储系统}

分布式对象存储系统(Distributed Object Storage System,简称DOSS)是一种新型的分布式存储系统,将数据以对象的形式存储在多个节点上,并通过对象标识符进行统一管理和访问。
它具有很好的可靠性、可用性、扩展性和数据处理能力\cite{kongeji},在互联网、金融、医疗、物联网等领域有广泛应用,是未来存储技术的重要发展方向。

\begin{figure}[h]
    \centering
    \includegraphics[width=0.8\textwidth]{my_figures/chapter2/存储模型对比图.png}
    \caption{对象存储模型与传统存储模型的区别}
    \label{fig:/存储模型对比图}
%     \note{注:图注的内容不宜放到图题中。}
\end{figure}

与传统的块存储和文件存储不同,DOSS具有以下特点:

(1)对象存储:DOSS将数据以对象的形式进行存储,一个对象通常包括数据、元数据和对象标识符等信息,方便管理和访问。

(2)分布式存储:DOSS采用分布式存储方式,将数据分散存储在多个节点上,提高了数据的可靠性和可用性。

(3)异构性:DOSS支持多种数据类型和格式,如文本、图片、视频等,同时也支持多种协议和接口\cite{kongqng2015keji}。

(4)高可靠性:DOSS采用副本机制和容错技术,保证数据的可靠性和安全性,即使某个节点出现故障,数据仍然可以被访问和恢复。

(5)高扩展性:DOSS采用分布式架构,可以根据实际需要添加或删除节点,从而实现扩展和收缩。

DOSS适用于需要存储海量数据的场景,同时具有很好的数据处理能力,支持数据分析、数据挖掘等操作,可以帮助企业更好地发掘数据价值。对象存储模型与传统存储模型的区别如图\ref{fig:/存储模型对比图}所示。




\section{分布式对象存储系统管控平台}

分布式对象存储系统(DOSS)管控平台是一种用于管理和控制DOSS集群的管理软件,它可以实现DOSS的统一管理、监控、调度和优化,提高DOSS的可用性、可靠性和性能\cite{kongqineji}。

DOSS管控平台通常包括以下几个方面的功能:

(1)集群管理:用于管理DOSS集群的拓扑结构、节点状态、数据分布等信息,支持节点的添加、删除、扩容、缩容等操作,实现集群的动态管理和调整。

(2)数据管理:用于管理DOSS存储的数据对象,包括对象的上传、下载、删除、复制、迁移等操作,支持数据的版本控制、回收站、检索等功能,提高数据的管理效率和可用性。

(3)性能监控:用于监控DOSS集群的性能和健康状况,包括节点的负载、网络延迟、数据吞吐量等指标,支持实时监控和历史数据的统计和分析,帮助管理员快速发现和解决故障。

(4)安全管理:用于管理DOSS集群的安全性,包括用户认证、访问控制、数据加密等功能,支持角色管理、审计日志等功能\cite{kongqii},保护DOSS存储的数据不被恶意访问或篡改。

(5)优化调整:用于优化DOSS集群的性能和资源利用率,包括节点负载均衡、数据分片调整、网络流量控制等功能,支持自动化调整和手动调整,提高DOSS集群的运行效率和稳定性。

总之,DOSS管控平台是一种重要的软件工具,可以帮助DOSS管理员管理和控制DOSS集群,提高DOSS的可用性、可靠性和性能,是DOSS技术的重要组成部分。

\section{AliIO}

AliIO是一种高性能、分布式的对象存储系统,与传统的存储以及其他类型的对象储存技术相较而言,其最大不同之处就是:它从一开始就按照针对性能需求更高的私有云规范完成了软
件架构工程设计,其简易架构如图\ref{fig:/AliIO简易架构图}所示。由于AliIO选择了更简易的方法来实现产品设计,它就可以完成传统对象储存技术所需要的所有功能,从而具备
了高效的特性,而且也不会因为更多的业务功能而影响AliIO的易用性和高效性。这样的设计结果所带来的最大优势就是:它可以更容易的完成原生对象储存业务,而且还具有弹性的伸
缩能力。

\begin{figure}[h]
    \centering
    \includegraphics[width=0.8\textwidth]{my_figures/chapter2/AliIO简易架构图.png}
    \caption{AliIO简易架构图}
    \label{fig:/AliIO简易架构图}
%     \note{注:图注的内容不宜放到图题中。}
\end{figure}

因此,AliIO在辅助存储,灾难修复以及归档等传统对象储存应用案例方面都体现的非常优秀。同时,它在机器学习、大数据分析、私有云、混合云等方面的对象储存技术上也独
具一格。当然,在大数据分析、高性能应用负载、原生云的支持等方面也发挥着巨大的优势。
其主要特性如下:

(1)高性能:拥有优异的读取特性,读出的速率高达185 GB/秒,而输入速率则高达175 GB/秒。对象存储可用作主内存层,主要用来管理Spark、Presto、TensorFlow、H2O.ai及其各类重复工作负载,而AliIO在云原生方面,具有比传统对象更高效的特性,能够适应云原生应用程序对高速吞吐和低延时的要求。

(2)可扩展性:采用了Web缩放器的设计,为数据存储创造了一个更轻松的缩放模式。AliIO从单个集群开始的拓展,这个集群能够通过和其他AliIO集群结合来产生全局名字空间,并且在使用中能够跨越几个不同的数据中心。而随着集群的加入,更多集群计算机也能够对名字空间进行拓展。

(3)云原生支持:和所有原生云计算技术的基本框架和构建过程相符合,把云计算技术的新技术和概念都加入到了其中,如支持Kubernetes、微服务和多租户的容器技术。

(4)简单易用:AliIO的指导性产品设计理念为极简主义。简化的设定方法降低了错误的机率,也增加了正常工作时间,从而提高了安全性,同时简易度也是系统稳定性的基石。配置与使用AliIO的流程都非常简洁,只需加载一个二进制软件然后运行,在数分钟内就能够实现。同时设置选项的数量也非常少,能够使设定错误的机率减至极少的程度,几乎不会出错。升级过程中也不需要使用简单的命令就可以实现,并且升级后不需要重启就能生效,极大降低了使用成本和运维成本。

(5)与Amazon S3兼容:亚马逊云的S3 API(接口协议)是在世界范围内达成共识的对象存储的协定\cite{kongq2eji},是全球认可的准则。




\section{微服务}


微服务架构是一种面向服务的架构风格,将应用程序拆分为一组松散耦合的、可独立部署和扩展的服务\cite{koyong2015keji}。每个服务都有自己的独立进程,可以使用不同的编程语言、数据库、技术堆栈等。微服务架构的优点包括:

(1)独立部署和扩展:每个服务都是独立的,可以单独部署和扩展,不会影响其他服务。

(2)松散耦合:服务之间的耦合度低,可以更容易地进行修改、重构和升级,不会影响整个系统。

(3)可组合性:每个服务都可以作为一个独立的组件,可以组合在一起构建更复杂的应用程序。

(4)容错性:如果一个服务失败,其他服务仍然可以正常运行,系统整体的容错性更强。

(5)独立团队开发:每个服务可以由不同的团队独立开发和维护,可以更快地推出新功能和修复问题。

本系统采取了微服务结构,把核心服务模块与非核心服务模块进行了分开,将系统各部分完全解耦,从而使得系统中整个的服务结构更加清晰,也减少了过多冗余的工作,维护起来也更加简单,从而使整个系统可以健康高速的发展迭代。

\section{后端开发框架}

(1)SpringBoot


Spring Boot是在 Spring框架的基础上发展而来\cite{konggyong2015keji},相比 Spring而言,SpringBoot省去了繁琐的配置过程,开发时快速高效。
Spring框架作为一个出色的开源应用架构,给后端开发人员带来了良好的解决方案,而控制反转也是它最大的特点所在,该特性
可以通过内置容器的设计思想对对象生命周期进行有效的管控。开发者可以使用Spring所规定的的格式撰写配置文件和配置类的
相关消息,将管理对象的任务由开发人员直接传递至Spring框架的内置容器\cite{kongqingyong201ji}。在程序启动后,开发者可以使用以下两类方法获得由
Spring托管的管理对象,一类方法是通过依赖注入,另一种是调用内置容器接口。本项目是使用Spring Boot的一个微服务。


(2)Spring Cloud Gateway

Spring Cloud中集成了多个框架,通过这些架构可以共同处理微服务架构中存在的各种问题\cite{kon2201ji},因此Spring Cloud为开发者提供了一个优秀而完备的解决方案,开发者也可以按照自身的实际需要应用这些模块。

\begin{figure}[h]
    \centering
    \includegraphics[width=0.8\textwidth]{my_figures/chapter2/网关在微服务中的位置图.png}
    \caption{网关在微服务中的位置图}
    \label{fig:/网关在微服务中的位置图}
%     \note{注:图注的内容不宜放到图题中。}
\end{figure}

Spring Cloud Gateway是Spring Cloud中最常见的一种功能,它的主要功能是协助开发者迅速而简单的建立一个多功能的API网关服务, API网关功能就相当于微服务的门户, API的转发和路由都可以在这个功能中实现,并且针对指定类型的API还可以按照需求进行一些鉴权或者限流的动作[23]。

而本项目的微服务 API 网关就是通过Spring Cloud Gateway框架进行构建的,先对用户的请求进行鉴权,然后再以请求的URI为基础把请求路由到特定的微服上。网关在微服务中的位置如图\ref{fig:/网关在微服务中的位置图}所示。

\section{辅助工具}

(1)Prometheus

Prometheus是一个开源的监控系统,用于收集、存储和查询各种类型的指标数据\cite{kongq5keji},例如系统性能指标、应用程序指标和服务健康状态等。它支持多种数据模型和查询语言,可以通过
HTTP接口提供数据,同时具有可视化和告警功能。此外,Prometheus还提供了一个丰富的可视化和告警功能\cite{kongqingy2015keji},可以帮助用户更好地理解系统和应用程序的运行状态,及时发现和解决问题。

(2)Grafana

Grafana是一个流行的开源数据可视化和监控平台,它允许用户根据所需的指标、时间范围和其他参数创建交互式的仪表盘和面板\cite{koing2keji},可以将数据从各种数据源(如Prometheus、
InfluxDB、Graphite等)中提取和可视化,也支持可扩展的插件和图形面板,包括时序数据、日志数据和各种应用程序指标。Grafana还具有用户管理、权限控制、警报通知等功能,
方便用户对系统进行全面监控和管理\cite{koi56keji}。它被广泛应用于云原生、DevOps、大数据、物联网等领域,是一个功能强大、易于使用的数据可视化和监控工具。

\section{本章小结}

本章首先介绍了分布式对象存储技术的组成和特性。之后阐述了本系统所采用的分布式对象存储系统AliIO的总体概况与关键技术特点,AliIO是由阿里云自行研发的一种高性能的分布式对象存储系统软件,是本系统设计的重要依据。接着介绍了分布式存储系统管控平台,它是一种用于管理和控制 DOSS 集群
的管理软件。最后引入了微服务框架、后端框架Spring Boot和SpringCloud Gateway,Spring Boot为程序员发布单个微服务带来了很大的方便,而SpringCloud Gateway则作为API网关很好的保护了微服务,同时也对与核心服务无关的系统维护功能进行隔离,从而使系统更加可靠且更易于维护。


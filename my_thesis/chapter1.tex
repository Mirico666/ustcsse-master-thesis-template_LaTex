% !TeX root = ../main.tex

\chapter{绪论}

\section{选题背景及研究意义}

近年来,自动驾驶、大数据、人工智能等技术都是研究热点,随之而来的是数据的爆炸式增长。
在此之前,数据都是采用集中存储的方式,这种方式起步较早,具有技术成熟、架构简单、稳定性好等特点,可以很好的支持高 IOPS、低延时和数据强一致性。此外,随着近年来全闪存阵列存储的迅速发展,IOPS 的性能与机械硬盘存储相比提高了 100 倍以上,这有效解决了 IOPS 的性能痛点。传统存储的系统架构的优点在于 I/O 路径短和访问延迟小,但其扩展能力有限,无法很好的支撑高并发的访问性能。随着大数据时代的来临,集中存储数据方式的增长空间逐渐受到限制,在系统的可靠性和安全性方面也面临着巨大的挑战,不能很好的服务于大规模存储应用。


% 在云计算的视角下,存储系统处于极其重要的位置。一方面,存储系统以逻辑卷和其他形式的块设备视图的形式向虚拟机提供存储虚拟化,用于引导和运行操作系统、数据库等重要基础软件;另一方面,存储系统也采用网络文件系统,以客户端访问容器和对象的
% 形式提供文件存储、大数据存储、云存储等多种服务。云计算对存储系统的性能、可靠性和可扩展性提出了越来越高的要求。大规模分布式存储系统的设计和优化将对云计算的服务质量产生巨大的影响。

% 分布式存储系统采用将数据分散成多个子集,将每个子集存储在相应的设备上。同时,分布式存储中利用了负载均衡的思想,同时也为了保证系用的扩展性,存储的职责不再由单一的设备承担,而是让多个设备同时发挥存储作用,数据被分散到多个独立的设备上进行存储,当系统获取信息时,会利用位置服务器对数据进行定位,让系统变得更加可靠,可用性和访问效率也得到了
% 提升,并具备了良好的扩展性能。基于对象的分布式存储(Object Storage)是近年来逐渐流行起来的一种新兴且可行的大规模存储解决方案。
% 它通过现有的存储组件、网络技术和处理技术实现了高扩展性和高吞吐量的目标。在对象存储的领域中,将文件转变划分为对象是重中之重,对象也是领域中的核心思想,每个对象都是不同的,具有唯一区分的标志。在常见的流程中,一般会先将文件划分为多个对象,然后将这些划分后的对象分散在整个存储设备的集群中进行存储,同时为了保证在控制文件访问权限时更加方便,管理文件的存储过程更加便利,分布式存储系统要将统一的存储空间提供给用户。而为了保证数据的安全性和避免出现数据丢失的问题,将每个对象进行冗余备份
% ,将这些备份存储在多个设备上。对象存储系统将数据块列表和对象列表进行映射,将各种数据块都简化成一个个对象,使系统的可扩
% 展性得到了大幅度提高,并且可以很轻松的对海量数据进行管理。

分布式存储系统通过分散存放的方法, 把数据分别存放到几个单独的存储器上。分布式网络存储系统的基本结构是可以扩展的, 储存负载也不再由一个服务器担当, 而是由多个服务器一起负担, 并通过位置服务器的定位储存信息, 使得整个信息系统看起来更为安全, 同时可用性和存取效能也获得了提高, 并具备了良好的扩展性能。基于面向对象的分布式储存系统 (Object Storage) 是近年来逐渐流行起来的一个新型且可行的大规模储存解决方案。它通过利用已有的存储器组件、网络技术和处理技术 达到了高扩展性和高吞吐量的目标。而面向对象则是对象存储的核心思想, 因为每个面向对象的标识都是独一无二的。对象存储将每个文档分割为几个对象, 然后将这些对象在整个的集群计算机中分散存放, 并将统一的对象存储提交给用户, 这样就能够更好的管理文件的存取权限和管理文件的存放流程了。为确保资料的安全并避免了数据丢失的问题, 为每个对象复制了多个备份, 将这些备份存储在多个设备上。对象存储系统通过数据块表与数据列表之间的映射, 把所有信息模块都简化为一个个对象, 使得数据库系统的可扩展性获得了提高, 同时也能够更容易地对海量信息资源实行集中式控制。

在信息技术高速发展的当今社会,数据已成为各企业都十分重视的竞争领域。随着数据量越来越多,人们对数据的存储和访问方面的需求日益强烈。所以, 建立起一种合适的储存管理机制是非常有价值的, 这种储存管理机制应该具有可综合利用存储设备、可对共享信息实现高效的保存与处理的特性, 同时具有高扩展度、高性能和高可用度的优点。因此,本文是在 AliIO 对象存储系统的上层设计和实现了一个智能化管理平台,为客户提供了多类型文件存储服务、直观化管理存储空间、多用户文件共享的文件存储管理系统。

\section{国内外研究现状}
\subsection{国外研究现状}
% 早在1980年,麻省理工学院(MIT)就开展了针对分布式对象存储的研究和实验SWALLOW,这也是世界上对于分布式对象存储的最早实现。而从1995年开始的
% 三年内,卡内基梅隆大学并行数据实验室一直致力于NASD(网络连接安全磁盘)项目的研究,该项目旨在构建一个安全、低延迟和高度
% 可扩展的存储系统,该系统通过身份验证和加密技术来保障系统的安全性,并不再由传统文件系统的文件接口而使用对象接口来进行
% 构建,NASD项目的成功成为了对象存储研究的里程碑,此后,NASD项目成为很多有关对象存储研究技术的基础。例如, Peter 和 Garth分别在这个项目的基础上完成了分布式文件系统的开发和实现,对应的文件系统Lustre 和 PanFS 也应运而生。

% 后来,亚马逊、谷歌等全球大型IT公司也开始了基于对象的存储技术的研究,并推动了其发展,分布式对象存储技术逐渐进入大众的视
% 野。谷歌和亚马逊先后推出了两款非常有代表性的作品,分别是GFS分布式文件系统和Dynamo分布式存储引擎。这两个分布式文件
% 系统基于不同的体系结构模式,实现了具有各自技术和特点的对象存储系统,得到了很多IT公司的青睐和借鉴。当今领先的基于对象的存储系统(如 GFS)主要针对少量大文件进行数据分析而优化。比如脸书的Haystack系统和推特的BlobStore系
% 统,为了存储大量小图像文件,都进行了不同程序的优化,而且还可以满足海量小文件的存储需求。2000年以后,先后开发了许多对象存储系统,如Swift、Amazon S3(标准存储服务)、HDFS(Hadoop分布式文件系统)、Ceph等。

% 2000年以来,大数据行业得到了快速发展,同时,对象存储技术也随之普及开来,在很多需要进行存储的地方都得到了应用。在开源项目中,有很多已经较为成熟的对象存储项目包括Ceph、Gluster以及OpenStack Swift等近年来持续升温,受到广泛关注,该技术的可行性和科学性得到了有力的证明和广泛的认可,对象存储技术
% 在各个领域得到了广泛的应用。在工业领域,自从云计算创始人Amazon于2006年提出S3对象存储平台之后,IT存储领域发生了彻底的改变。然而,单独使用商业公有云存储或开源框架显然无法满足那些追求相对安全性、可扩展性和灵活性的科研存储领域的需求,
% 这是因为目前的主流开源软件所能提供的功能与服务都比较简单和基础,而公有云所能提供的对象存储则在数据的安全性以及系统的可扩展性方面表现较差。

在一九八零年,麻省理工学院(MIT)的SWALLOW项目是世界最早的关于分布式数据存储的研究,这是分布式存储系统首次被实现。在一九九五年后的大三时起,卡内基梅隆的并行数据研究所就开始着力进行NASD(网络连接安全磁盘)方案的研发工作,该研究机构试图建立一种高安全性、少延时和高可使用性的存储系统,该体系通过利用身份验证和加密技术来保障了信息系统的安全性,并不再由单一存储介质的文件接口而是通过对象连接技术来实现了,NASD方案的实现就成了现代面向对象存储技术的重要标志,此后,NASD方案就成为了许多相关的存储技术领域的基石。例如,Lustre和PanFS就是由Peter和Garth分别基于这个方案而设计并实施的分布式文件系统。

后来,亚马逊、谷歌等世界主要信息技术企业都进行了基于数据的存储技术的探索,并促进了其成长,分布式对象存储技术逐渐进入大众的视野。谷歌和亚马逊也分别发布了二个相当有意义的产品,它们是GFS分布式操作系统和Dynamo分布式存储引擎。这二种分布式文件系统采用不同的结构方法,组成了具备相应技术与特性的文档存储系统,得到了很多 IT公司的青睐和借鉴。当今最领先的基于数据的存储系统(如GFS)主要根据少量的文件和数据分析来设计。如脸书的Haystack系统,和推特的BlobStore系统,对于存放大量的图像文件,都实现了不同程度的优化,同时也能够适应大量小文件的存放需要。二零零零年以后,中国先后发展了多种小文件存储系统,如Swift、Amazon S3(标准存储服务)、HDFS(Hadoop分布式文件系统)、Ceph等。

二零零零年至今,由于大数据领域的迅速兴起,对象存储技术已被广泛应用于各类大数据领域中。在开源平台方面,由Ceph、glu Ster、open Stack Swift等形成的面向对象存储底层存储方案在近年来也不断升温,其技术的可行性和科学性已经获得了有力的验证和社会普遍的认同,面向对象存储技术也在各个领域中获得了普遍的使用经验。在行业方面,自云计算技术开创者Amazon在二零零六年提出了s三对象存储模式以后,IT存储行业已经出现了根本的变化。但是,单纯采用的公共云存储的开源框架显然并不能适应所有要求的高安全、可扩展性和灵活性的科研数据方面的要求,因为开源软件的在内容服务方面较为稳定和简单,而公共云的在数据储存方面的安全和可扩展性方面则存在着缺陷。


\subsection{国内研究现状}

% 众所周知,在国内方面,阿里的淘宝文件系统TFS (Taobao File system)是目前最负盛名的分布式对象存储系统,
% 它具有高扩展性、高可用性和高性能的特点,主要是针对小文件存储需求而设计。由于淘宝后台存储的图片主要以8K以下的小图片
% 为主,所以TFS先将小文件合并成大文件,然后直接存储大文件,以此达到优化效果。

% 除了TFS以外,阿里巴巴旗下还有另外一个分布式对象存储系统,那就是阿里云对象存储服务OSS(Object Storage Service),它目前为许多企业用户和个人用户提供服务。 OSS具有海量、安全、高可靠的特点,适合存储图片、音视频、日志等海量文件。
% 各种终端设备、Web 程序和移动应用程序可以直接向 OSS 写入或读取数据。 借助 OSS,可以随时通过网络存储和访问各种结构化和非结构化数据文件,例如文本、图像和视频。
% OSS官网为用户提供了控制台界面和符合OSS规范的CLI控制命令,以供用户调用阿里云提供的APIS,向对象存储产品OSS中存储数据。 OSS中为每个用户都拥有一个存储空间Bucket,而系统中的每个文件都被认作一个相应的Object对象,每一个Bucket里包含多个Object,每一个Object对应一个文件,用户可以在 OSS 中创建属于自己的存储空间,用户创建
% 的Bucket数目可以是一个,也可以是多个,Bucket和Object的名称和属性都是可由用户自定义的。

% 此外,国内其他大型云存储服务提供商也开发出了对象存储领域的代表性产品,这些产品为用户提供了多备份、分布式低成本存储空间解决方案,
% 如腾讯云对象存储产品COS (cloud object storage)和金山云存储产品KS3(金山标准存储服务)。腾讯依托其
% 社交产品QQ和微信所拥有的庞大用户基础,为其推出了图片存储服务和视频云服务,并将优秀的图片和视频服务解决方案提供给企业和个人用户。KS3作为国内主要的对象存储产品之一,在云对象存储产品市场也同样占有一席之地,提供了优秀的数据存储和
% 视频存储解决方案。

众所周知,由淘宝网自己开发的淘宝操作系统TFS (Taobao File system)是中国国内目前最著名的分布式对象存储操作系统,它具备高扩展性、高可用性和高性能的优点,但主要是根据小文件存放需要而设计。由于淘宝后台存储的图片主要以8K以下的小图片为主,所以 TFS先将小文件合并成大文件,然后直接存储大文件,以此达到优化效果。

除TFS之外,阿里旗下还有另一种分布式对象存储系统,那便是阿里云对象储存业务OSS(Object Storage Service),它目前为许多企业用户和个人用户提供服务。OSS具备了海量、安全、高速可靠的特性,适于存放图像、音视频、日志等海量文件。所有的终端设备、Web程序和移动应用程序都能够直接从OSS输入或读取数据。通过OSS,人们能够随时使用网络存取并浏览所有结构化和非结构化的数据文档,例如文本、图像和视频。OSS官网还向开发者开放了控制台接口,以及遵循OSS标准的CLI管理命令,可让开发者通过调用由阿里云开发的APIS,在对象所存储的OSS中保存信息。OSS是给每位客户设定的空间Bucket,把每一位存储的文件都视为是一个Object,在一个Bucket里含有若干个Object,把每一位Object都相当于一个文件,客户也可在OSS中建立属于自己的空间,客户可以建立的空间Bucket数量既可以是一位,也可以是几个,空间Bucket和Object之间的命名方式和功能都是可以由客户定制的。

另外,中国国内的一些强大云存储服务提供商,还研制出了数据储存方面的代表性产品,通过此类产品给客户带来了多备份、分布式低成本存储解决方案,如腾讯云数据储存等产品COS (cloud object storage)和旧金山云存储产品KS3(金山标准存储服务)。腾讯公司凭借其社交产品QQ和微信所具备的强大用户基数,为其打造了照片存储业务和视频云业务,并把最优质的照片和视频业务解决方案提供给公司内部和个人用户。KS三作为业内最主要的对象存储产品之一,在中国云对象存储产品市场上也同样占据着一席之地,并创造了先进的数据存储与视频存储解决方案。

\section{本论文的主要工作}
本章中详细的介绍了一种基于阿里云自研AliIO架构的分布式数据存储处理平台的设计和实现,在开放的SDK的基础上基于微平台架构,设计了智能化的处理平台。本系统满足了个人用户和企业用户使用分布式对象存储的各项需求,包括文件的上传与下载,用户
访问权限控制,用户的执行策略控制,bucket的管理和服务器的管理等。本系统不仅能高效快速的执行文件的上传下载任务,实现在高并发环境下稳定运行,还易于扩展,将其他与存储有关的业务集成进本系统。本文的主要工作如下:

\begin{enumerate}
    \item 完成分布式存储管理系统的需求分析和概要设计。
    \item 基于业务需求对相关的数据库进行设计,理清个数据表之间的关联关系。
    \item 基于任务的具体要求和基于微平台的体系结构,把系统的所有模块分为具体的业务子单元。
    \item 实现各个功能模块,包括文件管理、策略管理、用户管理等模块的实现,
    \item 通过分析目前主导的分布式对象存储系统的结构技术,并总结多种框架的优点之
    处,并基于阿里云最新自研的新硬件AliFlash和分布式对象存储系统AliIO搭建系统。
\end{enumerate}



\section{本文的组织结构}

本文主要包括七个篇章,各个部分的内容编排如下:
第一章:绪论。对文章的技术背景、研究价值和分布式数据存储的国内研发情况分别加以说明,并介绍了文章的研究重点及其报告的结构设计。

第二章:相关技术简介。本文详尽阐述了对象存储器、对象存储系统的基本定义,及其当前主流的多种分布式文件操作系统。对上述分布式文件系统的特点进行了介绍,并加以了对比总结。

第三章:需求分析。本文将对分布式对象内存管理系统作出总体功能上的概括,从功能性与非功能性二种角度剖析分布式对象内存管理系统的总体需求。

第四章:概要设计。本文重点介绍了系统结构、数据库系统各功能的结构以及数据库系统概要设置等方面的知识点,通过借助数据库系统结构图与活动图来说明各功能设计的基本原理。

第五章:详细设计。本文重点阐述分布式数据存储处理技术的各个微业务的具体实现原理及其数据库的细节设置和应用。

第六章:系统测试与分析。本章重点针对整个软件系统展开全方位的软件测试,主要涉及实用性测试方法、安全性测试方法、集成测试等。

第七章:回顾与展望。本文重点在于对文章加以整理,研究其写作效果,总结存在的不足之处,并且预测下一步的发展。

\section{本章小结}
% 本章主要概括了论文的选题背景以及对分布式存储系统展开研究的意义,之后又详细阐述了目前对象存储技术的发展现状以及相关存储系统在国内外的研究进展。最后,对论文所做的主要工作和本文的组织结构展开详细介绍。
本文重点总结了本文的学术背景及其重要价值,介绍了对象存储技术与对象存储理论在国内的发展现状及其进展情况。最后,说明了作者要完成的主要任务以及文章的结构框架。
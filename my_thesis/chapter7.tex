\chapter{总结与展望}

本文重点对毕业论文工作加以总结,整理出本次毕业论文的主要优点和缺点,以及后续需要改进的地方,为后一阶段的工作提供指导。

\section{论文总结}

本论文主要是实现了一个分布式对象存储管理系统,该分布式对象存储基于阿里云自研软件AliIO,底层存储设备使用阿里云自研SSD
新硬件AliFlash ZNS,本论文主要任务是开发配套的管理系统,该系统同时服务于管理员和普通用户,普通用户通过该系统进行文件
的管理,管理员通过该系统进行管理。整个系统采用微服务设计的思想,将各部分设计成单独的模块,充分解耦,使系统更加安全稳定。

本文的主要工作如下:

\begin{enumerate}
    \item 完成分布式存储管理系统的需求分析和概要设计。
    \item 基于业务需求对相关的数据库进行设计,理清个数据表之间的关联关系。
    \item 基于项目的现实需要和基于微服务的架构,把一个信息系统的所有职能划分为一定的职能子模块。。
    \item 实现各个功能模块,包括文件管理、策略管理、用户管理等模块的实现,
    \item 通过研究主流的分布式数据存储系统的结构方式,总结多种模式的优点之处,并基于阿里云最新自研的新硬件AliFlash和分布式对象存储系统AliIO搭建系统。
\end{enumerate}

\section{问题和展望}

目前,随着数据的日益增多,人们需要存储的文件也在逐渐增多,这使得人们对存储系统的需求越来越大,分布式对象存储可支持多类型文件
的上传,速度快,容量大,最重要的是安全性好。本论文中基于AliIO构建的分布式对象存储系统可以满足大部分人的需求,在该系统上
开发一个管理平台可以方便用户使用存储系统,也方便管理人员进行维护和管理。在以后随着用户的增多,可能需要对系统进行扩容,或者
可能会对系统进一步的优化,以适应新的需求。